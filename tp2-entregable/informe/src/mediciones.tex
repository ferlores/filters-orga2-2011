
\subsection{Mediciones}
Una vez desarrolados los filtros tanto en \C \  como en \ass, necesitabamos par'ametros de comparaci'on
para ver si efectivamente nuestras conjeturas iniciales eran ciertas o no.
Adem'as de correr los test solicitados por la c'atedra se generaron casos de prueba,
especificamente pensados para poner a prueba los algoritmos en los casos borde y en algunos casos
favorables.\\
\subsection{An'alisis de factores}
Antes de generar los casos de prueba, analizamos con que tipo de im'agenes nuestros algoritmos pod'ian
llegar a comportarse una una manera inusual. Como conjetura inicial establecimos que no deberia haber 
fluctuaciones en el las funciones desarrolladas en \C \ ya que el recorrido de la matriz es lineal y
de a un elemento por vez, con lo cual las dimensiones no deber'ian ser un factor relevante. \\
Para el caso de las funciones en \ass, creemos que los principales factores que pueden afectar la 
performance son : \\
\begin{itemize}
 \item{Cantidad de accesos a memoria}.
 \item{Cantidad de instrucciones de movimiento interno de bits en registro}
 \item{Dimensiones de la matriz,(par'ametro de entrada)}
\end{itemize}
El primer factor pudo subsanarse construyendo un c'odigo m'as eficiente que su version inicial, 
usando instrucciones y direccionamientos adecuados(que permitieran ganar m'as performance).
El segundo factor pudo reducirse en gran parte con el uso de m'ascaras, sin embargo en algunos casos
no se pudieron sacar ya que sin ellos no pod'ia obtenerse la funcionalidad deseada.\\
El tercer factor esta determinado por la entrada y el algoritmo, se explica en la secci'on siguiente.
\subsection{Estableciendo Criterios}
En primer lugar se estableci'o como criterio de cantidad de pixeles los mas aproximado posible
a 480000.\\
Como se menciona anteriormente el algoritmo del filtro es el que va determinar que casos son borde
y en cuales se puede tener una performance 'optima. Para ello, es necesario diferenciar  dos tipos de
filtro: Los que van de color a escala de grises( monocromatizar,separar canales) y en blanco y negro
(umbralizar,invertir,normalizar,suavizar). \\
Para los que toman imagen en color,en el ciclo principal si bien las lecturas son de 16 bytes se procesa 
de a 5 pixeles por vez, con lo cual si la cantidad de pixeles de ancho es multiplo de 5 el algoritmo 
realizar'ia la cantidad justa de lecturas en memoria, en este caso deberia consumir menos
ciclos de reloj. 
En el caso contrario en que la anchura no se multiplo de 5 el algoritmo debe realizar una lectura
mas de la prevista en el final de cada ciclo que recorre columnas, con lo cual deberia consumir
mas ciclos de reloj, mas aun si la cantidad de pixeles alto aumenta, el costo de esas lecturas extra por 
columna se har'ia evidente en una eventual medici'on.
Teniendo en cuenta esto se decidi'o medir con im'agenes de la siguiente distribuci'on de pixeles:
 \begin{itemize}
 \item{800x600}.
 \item{10x48000}
 \item{11x43636}
\end{itemize}
Para los filtros que trabajan im'agenes blanco y Negro ocurre algo similar a lo anterior solo que cada
pixel esta representado por un byte por lo tanto por cada lectura se procesa de a 16 pixeles. De forma
an'aloga asociamos mejor performance con las im'agenes cuyo anchura en pixeles es un multiplo de 16, y 
una peor performance con las que son coprimas con 16, ya que necesitara una lectura mas por cada columna
de la matriz, mas a'un procesaran algunos datos dos veces. \\
Los \textit{sizes} elegidos para estos caso fueron:\\
\begin{itemize}
 \item{800x600}.
 \item{16x30000}
 \item{17x28235}
\end{itemize}
\subsection{Opciones descartadas}
En un momento se penso ver cual era el rendimiento de los algoritmos con una distribuci'on 
aleatoria de im'agenes, pero decidimos omitirla debido a que era muy costosa y no nos 
daria informaci'on valiosa acerca del comportamiento ya que los \textit{sizes} variables pueden comportarse 
de manera an'aloga por la forma de procesar los datos y no habria una diversidad de comportamiento.\\
Se decidio enfocar los casos de prueba en los \textit{sizes} de im'agen y no tanto en el contenido de 
sus pixeles, vale decir no se buscaron especificamente im'agnes con muchos blancos, grises o negros.   


