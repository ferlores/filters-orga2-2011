\documentclass[10pt,a4paper]{article}
\usepackage[latin1]{inputenc}
\usepackage{amsmath}
\usepackage{amsfonts}
\usepackage{amssymb}
\usepackage{array} 
\usepackage{tabularx}
\usepackage[spanish, activeacute]{babel}
\usepackage[small]{caption}
\usepackage[txt]{caratula}
\usepackage{a4wide}

\makeatletter
\newcommand\textsubscript[1]{\@textsubscript{\selectfont#1}}
\def\@textsubscript#1{{\m@th\ensuremath{_{\mbox{\fontsize\sf@size\z@#1}}}}}
\newcommand\textbothscript[2]{%
  \@textbothscript{\selectfont#1}{\selectfont#2}}
\def\@textbothscript#1#2{%
  {\m@th\ensuremath{%
    ^{\mbox{\fontsize\sf@size\z@#1}}%
    _{\mbox{\fontsize\sf@size\z@#2}}}}}
\def\@super{^}\def\@sub{_}

\catcode`^\active\catcode`_\active
\def\@super@sub#1_#2{\textbothscript{#1}{#2}}
\def\@sub@super#1^#2{\textbothscript{#2}{#1}}
\def\@@super#1{\@ifnextchar_{\@super@sub{#1}}{\textsuperscript{#1}}}
\def\@@sub#1{\@ifnextchar^{\@sub@super{#1}}{\textsubscript{#1}}}
\def^{\let\@next\relax\ifmmode\@super\else\let\@next\@@super\fi\@next}
\def_{\let\@next\relax\ifmmode\@sub\else\let\@next\@@sub\fi\@next}
\makeatother

\newcolumntype{C}{>{\footnotesize\centering\arraybackslash}X}

\newcommand{\xmmW}[2]{
\fontfamily{pcr}\selectfont	
\begin{center}
\begin{tabularx}{400pt}{@{}p{60pt}|C|C|C|C|C|C|C|C|l}
\cline{2-9}
#1 & #2 &\\
\cline{2-9}
\multicolumn{1}{r@{}}{^{\emph{128}}} & \multicolumn{8}{c}{\textit{\scriptsize(word packed)}} & \multicolumn{1}{@{}l}{^{\emph{0}}}\\
\end{tabularx}
\end{center}
}

\newcommand{\xmmD}[2]{
\fontfamily{pcr}\selectfont	
\begin{center}
\begin{tabularx}{400pt}{@{}p{60pt}|C|C|C|C|l}
\cline{2-5}
#1 & #2 &\\
\cline{2-5}
\multicolumn{1}{r@{}}{^{\emph{128}}} & \multicolumn{4}{c}{\textit{\scriptsize(double-word packed)}} & \multicolumn{1}{@{}l}{^{\emph{0}}}\\
\end{tabularx}
\end{center}
}


% Fancy Header
\usepackage{fancyhdr}
\fancyhf{}
\fancyhead{}
\lhead{Seccion} % texto izquierda de la cabecera
\rhead{\Titulo} % número de página a la derecha
\lfoot{} % texto izquierda del pie
\rfoot{\thepage} 
\pagestyle{fancy}


% Carátula
\materia{Organizaci\'on del Computador II}
\submateria{Segundo Cuatrimestre del 2011}
\titulo{Trabajo Pr\'actico N\'umero 2}
\subtitulo{Filtros de im\'agen}
\integrante{Celave, Martin}{530/08}{tolacelave@gmail.com}
\integrante{Colombo, Ricardo Gaston}{156/08}{ricardogcolombo@gmail.com}
\integrante{Lores, Fernando}{718/01}{ferlores@gmail.com}
\grupo{Lider espiritual: Angiulino "El chino" Cubrepileta}

\begin{document}

\maketitle


\section{Introducci'on}

El objetivo de este informe es describir el proceso de como la aplicacion algoritmica de filtros sobre imagenes
puede ser optimizada notablemente(en comparaci'on con ANSI \C) si se aprovecha el modelo de instrucciones SIMD. 

Para ello implementamos siete algoritmos en \C\ y en \ass. Los mismos recorren las im'agenes aplicando diferentes c'alculos. Las implementaciones de alto nivel son bastantes triviales y directas gracias a la posibilidad que nos brinda el lenguaje para acceder a las distintas posiciones de memoria. Sin embargo la gran desventaja que tenemos aqui es que accedemos y procesamos los datos de a uno. En cambio, en la implementaci'on de bajo nivel, si bien podemos acceder a una gran cantidad de datos por vez, el acceso a ellos hace que los algortimos a veces se vuelvan ingeniosos. 

En la siguiente secci'on explicaremos como encaramos la implementaci'on de los filtros pedidos. Primero describiremos la estrategia que elegimos para iterar las im'agenes junto con sus variaciones. Luego nos abocaremos a describir la mec'anica que utilizamos en cada algoritmo para procesar los datos. Como dijimos antes, las implementaciones en alto nivel son bastantes directas, por lo que no nos detendremos mucho explicandolas. El lector puede remitirse driectamente al c'odigo para un mayor detalle de los mismos.

En la secci'on \ref{sec:mediciones} realizaremos mediciones de las distintas implementaciones sobre un conjunto de imagenes dado, comparando y analizando los resultados obtenidos. Es claro inmaginar que deber'ia haber una  gran diferencia en cuanto \textit{performance} y esperamos a lo largo de este informe verificar dicha conjetura. Sin embargo tambien veremos que existen casos menos favorables que otros donde la mejora no es tan grande comparada como en otros casos. 

Por 'ultimo en la secci'on \label{sec:conclusiones} recapitularemos los resultados obtenidos, analizando la relaci'on costo-beneficio de las distintas implementaciones. Ademas comentaremos las principales dificultades encontradas y otras experiencias que obtuvimos durante la realizaci'on de este trabajo pr'actico


\include{ciclos}
\subsection{Monocromatizar}
\subsubsection{Monocromatizar $\epsilon = 1$ }

\begin{figure}[hb]
\xmmW{shift }{B_5R_4 & G_4B_4 & R_3G_3 & B_3R_2 & G_2B_2 & R_1G_1 & B_1R_0 & G_0B_0}
\xmmD{shift }{0G_2 0B_2 & R_1G_1 & B_1R_0 & G_0B_0}
\caption{Estrat'egia de normalizar\_asm}
\label{est:m-uno}
\end{figure}


\subsubsection{Monocromatizar $\epsilon = \infty $ }

\subsection{Separar canales}
El objetivo de este filtro es generar tres im'agenes distintas en escala de grises. Cada imagen debe contener en cada p'ixel lo correspondiente al valor de la capa que estamos separando del p'ixel original. La implementac'on en \C\ es trivial, ya que puedo direccionar a cada valor del p'ixel que estoy recorriendo de una manera directa\footnote{para m'as informaci'on de la implementaci'on de la funci'on remitirse al c'odigo fuente}. por otra parte, en \ass\ debemos aislar cada valor RGB de manera que posteriormente podamos manipularlos. En general esto se hace con instrucciones de desempaquetado. Sin embargo en nuestro caso preferimos utilizar otra estrategia. La idea es realizar desplazamientos (empaquetados de words) a izquierda y derecha para desempaquetar todos los valores. En la figura \ref{est:separar-1} podemos observar como con un desplazamiento a derecha de un byte (linea b) podemos desempaquetar ciertos datos, mientras que con un desplazamiento primero a izquierda y luego a derecha, logramos separar los valores que estaban faltando (linea c).

\begin{figure}[ht]
\xmmW{a) src[15$*$i]}{XR_4 & G_4B_4 & R_3G_3 & B_3R_2 & G_2B_2 & R_1G_1 & B_1R_0 & G_0B_0}
\xmmW{b) psrlw 1}{0X& 0G_4 & 0R_3 & 0B_3 & 0G_2 & 0R_1 & 0B_1 & 0G_0}
\xmmW{c) psllw 1, psrlw 1}{0R_4 & 0B_4 & 0G_3 & 0R_2 & 0B_2 & 0G_1 & 0R_0 & 0B_0}
\caption{Estrategia de desempaquetamiento im'agenes a color}
\label{est:separar-1}
\end{figure}

Una vez desempaquetados estos datos, preparamos m'ascaras para colocar ceros en los words que no necesitamos. Si prestamos atenci'on en como quedan conformados los registros desempaquetados, notamos que a veces vamos a necesitar filtrar tres words y a veces dos words. Es por eso que solo dos mascaras (desplazandolas a izquierda y derecha) fueron suficientes para separar todos los B, luego los R y por 'utimo los G. Las m'ascaras estan compuestas por words de \textit{unos} donde estan los datos que queremos y ceros donde estan los valores que no necesitamos en este momento. En la figura \ref{est:separar-2} podemos observar como quedan los registros despues de utilizar las m'ascaras para filtrar los valores G (lineas d y e) y como queda combinado en un solo registro todos los valores que necesitamos (linea f).

\begin{figure}[ht]
\xmmW{d) pand b, mask_1}{00 & 00 & 0R_3 & 00 & 00 & 0R_1 & 00 & 00}
\xmmW{e) pand c, mask_2}{0R_4 & 00 & 00 & 0R_2 & 00 & 00 & 0R_0 & 00}
\xmmW{f) por d, e}{0R_4 & 00 & 0R_3 & 0R_2 & 00 & 0R_1 & 0R_0 & 00}
\caption{filtrado y combinaci'on usando m'ascaras \sc{\scriptsize 0000FF0000FF0000} y \sc{\scriptsize FF0000FF0000FF00}}
\label{est:separar-2}
\end{figure}

Ahora resta reacomodar los datos para poder escribirlos en memoria. Para esto usamos las instrucciones de intercambio que el procesador nos ofrece: \textit{pshufd, pshuflw y pshufhw}. Con estas tres instrucciones es posible realizar las tres funciones mas comunes de intercambio a nivel de words: \textit{broadcast}, \textit{swap}, \textit{reverse}\footnote{Intel^{\copyright} 64 and IA-32 Architectures Optimization Reference Manual, secci'on 5.4.11}. La figura \ref{est:separar-3} muestra como reacomodamos los words (lineas g y h) y empaquetamos para obtener en los bytes menos significativos los cuatro p'ixeles ya procesados (linea i). En este punto solo nos queda escribir en la imagen destino esos 4 bytes, desplazar el registro SSE de modo que el quinto valor quede en la parte menos significativa, y tambi'en escribirlo. Cabe destacar que para hacer esta 'ultima escritura ultilizamos un registro de prop'osito general, porque no existe una instrucci'on que nos permita copiar un byte determinado del registro SSE a memoria.

\begin{figure}[ht]
\xmmW{g) pshuflw f, (3,0,2,1)}{0R_4 & 00 & 0R_3 & 0R_2 & 00 & 00 & 0R_1 & 0R_0}
\xmmD{h) pshufd g, (3,1,2,0)}{0R_4  00 & 00  00 & 0R_3  0R_2 & 0R_1  0R_0}
\xmmW{i) packuswb h }{xx & xx & xx & xx & 0R_4 & 00 & R_3R_2 & R_1R_0}
\caption{\textit{swap} de words y double-words}
\label{est:separar-3}
\end{figure}

El proceso de los otros canales son similares. Siempre utilizamos la misma estrategia: separamos los valores, los enmascaramos, los combinamos, y por ultimo realizamos los intercambios necesarios para poder bajarlos a memoria. Para mayor detalle de como procesamos los canales B y G remitirse a los comentarios del c'odigo adjunto.



\subsection{Umbralizar}
Este filtro proproduce im'agenes con solamente tres colores (blanco, gris y negro), dependiendo del valor que tenga cada p'ixel.
El procesamiento de cada p'ixel esta definido por la siguiente funci'on:

$$I_{out}(p) = \left\{\begin{array}{lcc} 0 & \text{si } p \leq umbralMin\\  128 & \text{si } umbralMin < p \leq umbralMax \\ 255 & \text{si } p > umbralMax \end{array} \right. $$

La implementaci'on en \C\ es nuevamente trival. Recorremos, como siempre en estos casos, secuencialmente de a un byte mientras que el procesamiento de cada p'ixel se reduce a dos comparaciones para elegir el valor correcto del resultado.

En cambio, para la implementaci'on en \ass\ procesaremos de a 16 bytes, y no tendremos que desempaquetar los valores

tenemos varios pasos. Primero vimos que para realizar las comparaciones era necesario tener los umbrales empaquetados(de a bytes) en los registros de 128 bits. Para esto cargamos los valores que recibimos como par'ametro de entrada en un registro de proposito general. Luego realizamos desplazamientos de a de 8 bits a la izquierda y sucesivas sumas del valor hasta que conseguimos un doble \textit{word} lleno. Despues movemos a dos registros de 128 bits el registro de prop'osito general con el umbral m'inimo y m'aximo y utilizamos la instruccion de intercambio \textit{shufle} de double words para replicar el valor a lo largo de todo el registro SSE. 

Por otro lado tambi'en necesitabamos un registro que contenga el n'umero 128, que representa el gris en la im'agen.
Para obtenerlo realizamos el mismo procedimiento que hicimos con los valores m'inimos y m'aximos, solo que el valor inicial lo obtenemos a aprtir de la constante 8080h\footnote{el valor 80h corresponde al valor 128 en decimal} .

El resultado de este proceso se ve en la figura \ref{est:u-uno}.

\begin{figure}[h!]
\xmmW{umbral m'inimo}{Umin & Umin & Umin & Umin & Umin & Umin & Umin & Umin}
\xmmW{umbral m'aximo}{Umax & Umax & Umax & Umax & Umax & Umax & Umax & Umax}
\xmmW{gris}{128 & 128 & 128 & 128 & 128 & 128 & 128 & 128}
\caption{constantes necesarias para el proceso del algoritmo}
\label{est:u-uno}
\end{figure}


Una vez ya definidos los umbrales en los registros comenzamos a procesar los datos. Para saber cuales son los elementos que estan por
sobre el umbral hacemos una resta saturada entre el registro que contiene el umbral m'aximo y los valores originales de la im'agen. De esta manera nos quedan ceros en los valores que eran mayores o iguales al m'aximo y valores distintos a cero en las otras posiciones. Entonces utilizamos 
la instrucci'on \textit{pcmpeqb} para armar una m'ascara a partir de la comparaci'on de este resultado contra un registro lleno de ceros, obteniendo por resultado Fh en las posiciones donde el valor original era mayor al umbral y ceros en las otras posiciones.

Ahora realizo un procedimiento similar. Hacemos una resta saturada es entre el valor original y el valor m'inimo, dando por resultado un registro con ceros en los bytes que estan debajo del umbral m'inimo (nuevamente gracias a la saturacion). Luego, utilizando la instrucci'on \textit{pcmpeqb} contra un registro lleno de ceros, armamos otra m'ascara con Fh en los bytes que estan debajo del umbral m'inimo.

Luego combinamos las dos m'ascaras con un \textit{por} y al resultado le aplicamos la instrucci'on \textit{pcmpeqb} con
un registro con ceros. Esta 'ultima operacion es, en este caso, similar a una negacion bit a bit. Por lo que obtenemos Fh donde hab'ia antes ceros, es decir, en las posiciones donde se encuentran los valores medios de la imagen.

Por 'ultimo, utilizamos esta 'ultima m'ascara junto con el registro que contiene la constante 128 para realizar un \textit{pand} y obtener como resultado un registro con los grises en sus posiciones correspondientes. Por 'ultimo solo queda colocar blancos en las posiciones donde la imagen es mayor al umbral. Recordamos que tenemos calculada una mascara con Fh en esas posiciones y ceros en las dem'as, por lo que un simple \textit{por} combinaria este registro el con registro con grises, obteniendo los bytes necesarios para escribirlos en la nueva imagen.
\subsection{Invertir}
El fin de invertir es que dado una imagen en escala de grises se le resta el valor maximo de un pixel al valor que esta en el pixel, de esta manera siempre se resta el valor 255 a el valor del pixel.

Con lo cual lo que realizamos es levantar el valor de los pixeles en un registro sse , el xmm0, y a un registro le aplicamos la instruccion pcmpeqb , con lo cual el registro queda completo de 0xF en el registro xmm7.
Luego restamos con saturacion al registro xmm7 el valor de los pixeles de la imagen, el registro xmm0, asi de esta manera obtenemos el valor deseado, la resta es con saturacion ya que si no se produce overflow.

\begin{figure}[ht]
\xmmW{paso 7.}{FF & FF & FF & FF & FF & FF & FF & FF & FF & FF & FF & FF & FF & FF & FF & FF}
\xmmW{paso 0.}{a_15 & a_14 & a_13 & a_12 & & a_11 & a_10 & a_9 & a_8 & a_7 & a_6 & a_5 & a_4 & a_3 & a_2 & a_1 & a_0}
\caption{Los registros antes de ser procesados}
\end{figure}

\begin{figure}[ht]
\xmmW{paso 0.}{FF - a_15 &  FF - a_14 & FF - a_13 & FF - a_12 & FF - a_11 & FF - a_10 & FF - a_9 & FF - a_8 & FF - a_7 & FF - a_6 & FF - a_5 & FF - a_4 & FF - a_3 & FF - a_2 & FF - a_1 & FF - a_0}
\caption{El regustro resultante}
\end{figure}

Luego de procesado los datos guardamos el resultado en la imagen destino.

\subsection{Suavizado Gaussiano}

\subsubsection{Iteracion}
Primero tuvimos que tener en cuenta ciertas limitaciones en la el calculo de los pixeles que nos impon'ia la 
misma m'ascara, ya que al ser una m'ascara de 3x3 y calcular el pixel del centro, en los bordes no es posible
 ya que los pixeles no tienen entorno.
Por otro lado, en ciclo se leen mas de 16 pixeles, a diferencia de los otros algoritmos, 
ya que al realizar 3 lecturas moviendose de a 1 pixel se estan levantando 18 y se procesan solo  
15, esto a su vez nos indica que al llegar al final de una fila debemos 
tener en cuenta la cantidad de p'ixeles que procesamos con lo cual debemos preguntar si estamos 16 posiciones 
antes del fin de la fila.

\subsubsection{Algoritmo de proceso de datos}
El objetivo de este filtro es obtener una imagen resultante con reducci'on de ruido y difuminaci'on. 
Para ello utilizaremos el m'etodo de aplicaci'on de m'ascaras, con lo cual el p'ixel resultante se obtiene con la suma
de los valores de los p'ixeles del entorno multiplicados por la m'ascara.

La m'ascara que utilizaremos sera la siguiente(tabla \ref{tab:s-uno} ):

\begin{table}[h!]
\begin{center}
\begin{tabular}{| c | c | c |}
\hline
1/16 & 2/16 & 1/16 \\ \hline
2/16 & 4/16 & 2/16 \\ \hline
1/16 & 2/16 & 1/16 \\ \hline
\end{tabular}
\end{center}
\caption{Mascara a aplicar}
\label{tab:s-uno}
\end{table}

Con lo cual el p'ixel resultante sera el p'ixel que se encuentra en el centro del entorno, y el mismo tendra el siguiente valor:
$$
Iout(i,j) /= I_{in}(i-1,j-1) \cdot{} 1/16 + I_{in}(i-1,j) + I_{in}(i-1,j+1) \cdot{} 1/16 + I_{in}(i+0,j-1) \cdot{} 2/16 + I_{in}(i+0,j)  \cdot{} 4/16 +  I_{in}(i+0,j+1)  \cdot{} 2/16 + I_{in}(i+1,j-1)  \cdot{} 1/16 + I_{in}(i+1,j) \cdot{} 2/16 + I_{in}(i+1,j+1)  \cdot{} 1/16
$$
La implementaci'on en C es trivial teniendo la formula mensionada anteriormente.

Por el lado de la implementacion en \ass, al ver que cada p'ixel deb'ia ser la suma de su entorno y 
que nosotros trabajar'iamos con los registros \textit{sse} de 128 bits, decidimos 
dividir en 3 partes el algoritmo para tratar cada una de las lineas de la matriz resultante.

Comenzamos el algoritmo realizando una lectura de 16 bytes a partir de un puntero base(no es el base pointer), 
luego realizamos una segunda y tercer lectura(para guarda informaci'on del entorno)
desplazando el puntero base en uno y dos bytes respectivamente.
La siguiente im'agen muestra esas lecturas(figura \ref{est:s-dos} ):

\begin{figure}[h!]
\xmmW{paso 1.}{a_15a_14 & a_13a_12 & a_11a_10 & a_9a_8 & a_7a_6 & a_5a_4 & a_3a_2 & a_1a_0}
\xmmW{paso 2.}{a_16a_15 & a_14a_13 & a_12a_11 & a_10a_9 & a_8a_7 & a_6a_5 & a_4a_3 & a_2a_1}
\xmmW{paso 3.}{a_17a_16 & a_15a_14 & a_13a_12 & a_11a_10 & a_9a_8 & a_7a_6 & a_5a_4 & a_3a_2}
\xmmW{paso 7}{FF & FF & FF & FF & FF & FF & FF & FF}
\caption{Levantado de datos de la primer linea}
\label{est:s-dos}
\end{figure}

Luego desempaquetamos los datos para operar como words, con lo cual primero copiamos los registros que 
tenemos con los datos para operar todos los p'ixeles que levantamos, ya que al desempaquetar los mismos 
quedaran words con datos donde su parte alta es cero.
Ya copiados los registros procedimos a desempaquetarlos, previamente habiendo dejado un registro 
lleno con ceros(contra el cual hacemos el desmpaquetado) quedando as'i figura \ref{est:s-tres}.

\begin{figure}[h!]
\xmmW{reg 4.}{0 a_15 & 0 a_14 & 0 a_13 & 0 a_12 & a_11 & 0 a_10 & 0 a_9 & 0 a_8}
\xmmW{reg 1.}{0 a_7 & 0 a_6 & 0 a_5 & 0 a_4 & 0 a_3 & 0 a_2 & 0 a_1 & 0 a_0}
\xmmW{reg 5.}{0 a_16 & 0 a_15 & 0 a_14 & 0 a_13 & 0 a_12 & 0 a_11 & 0 a_10 & 0 a_9}
\xmmW{reg 2.}{0 a_8 & 0 a_7 & 0 a_6 & 0 a_5 & 0 a_4 & 0 a_3 & 0 a_2 & 0 a_1}
\xmmW{reg 6.}{0 a_17 & 0 a_16 & 0 a_15 & 0 a_14 & 0 a_13 & 0 a_12 & 0 a_11 & 0 a_10 }
\xmmW{reg 3.}{0 a_9 & 0 a_8 & 0 a_7 & 0 a_6 & 0 a_5 & 0 a_4 & 0 a_3 & 0 a_2}
\xmmW{reg 7}{FF & FF & FF & FF & FF & FF & FF & FF}
\caption{los registros 1,2,3 corresponden a partes bajas y los 4,5,6 a partes altasAsm}
\label{est:s-tres}
\end{figure}
Luego se utiliza la operaci'on \textit{psllw}, que ser'ia equivalente a multiplicar por 2, en los 
registros de la segunda lectura y sumamos las partes bajas guard'andolas en los acumuladores.
Ya procesada la primer linea de la m'ascara procedemos a la segunda, para la cual sumamos el ancho de la 
im'agen en 
p'ixeles para pararnos en la segunda fila de la matriz, luego con el mismo m'etodo copiamos 
y desempaquetamos,  
usando los mismos registros que en la primer linea(ya que podemos reutilizarlos debido a que los datos est'an
guardados), a diferencia que esta vez se multiplican por 2 los 
registros de la primera y tercer operacio'n de lectura por otra parte los registros correspondientes a 
la segunda lecutra se multiplican 4, luego sumamos estos a los acumuladores correspondientes.

Por 'ultimo pasamos a la tercer linea de la misma manera que la anterior, realizamos tres lecturas a memoria
cada una desplazada en un byte y realizamos el mismo proceso que en la primer linea ya que 
las operaciones que deben hacerse son las mismas(ver tabla \ref{tab:s-uno}) copiando, desempaquetando 
y multiplicando, para luego sumar en los acumuladores correspondientes a la parte alta y parte baja, y dividirlos por 16.
Al finalizar de sumar todas las lineas procedemos al empaquetado de la parte alta con la parte baja y 
guardamos los datos.




\subsection{Normalizar}
%
\subsection{Mediciones}
Una vez desarrolados los filtros tanto en \C \  como en \ass, necesitabamos par'ametros de comparaci'on
para ver si efectivamente nuestras conjeturas iniciales eran ciertas o no.
Adem'as de correr los test solicitados por la c'atedra se generaron casos de prueba,
especificamente pensados para poner a prueba los algoritmos en los casos borde y en algunos casos
favorables.\\
\subsection{An'alisis de factores}
Antes de generar los casos de prueba, analizamos con que tipo de im'agenes nuestros algoritmos pod'ian
llegar a comportarse una una manera inusual. Como conjetura inicial establecimos que no deberia haber 
fluctuaciones en el las funciones desarrolladas en \C \ ya que el recorrido de la matriz es lineal y
de a un elemento por vez, con lo cual las dimensiones no deber'ian ser un factor relevante. \\
Para el caso de las funciones en \ass, creemos que los principales factores que pueden afectar la 
performance son : \\
\begin{itemize}
 \item{Cantidad de accesos a memoria}.
 \item{Cantidad de instrucciones de movimiento interno de bits en registro}
 \item{Dimensiones de la matriz,(par'ametro de entrada)}
\end{itemize}
El primer factor pudo subsanarse construyendo un c'odigo m'as eficiente que su version inicial, 
usando instrucciones y direccionamientos adecuados(que permitieran ganar m'as performance).
El segundo factor pudo reducirse en gran parte con el uso de m'ascaras, sin embargo en algunos casos
no se pudieron sacar ya que sin ellos no pod'ia obtenerse la funcionalidad deseada.\\
El tercer factor esta determinado por la entrada y el algoritmo, se explica en la secci'on siguiente.
\subsection{Estableciendo Criterios}
En primer lugar se estableci'o como criterio de cantidad de pixeles los mas aproximado posible
a 480000.\\
Como se menciona anteriormente el algoritmo del filtro es el que va determinar que casos son borde
y en cuales se puede tener una performance 'optima. Para ello, es necesario diferenciar  dos tipos de
filtro: Los que van de color a escala de grises( monocromatizar,separar canales) y en blanco y negro
(umbralizar,invertir,normalizar,suavizar). \\
Para los que toman imagen en color,en el ciclo principal si bien las lecturas son de 16 bytes se procesa 
de a 5 pixeles por vez, con lo cual si la cantidad de pixeles de ancho es multiplo de 5 el algoritmo 
realizar'ia la cantidad justa de lecturas en memoria, en este caso deberia consumir menos
ciclos de reloj. 
En el caso contrario en que la anchura no se multiplo de 5 el algoritmo debe realizar una lectura
mas de la prevista en el final de cada ciclo que recorre columnas, con lo cual deberia consumir
mas ciclos de reloj, mas aun si la cantidad de pixeles alto aumenta, el costo de esas lecturas extra por 
columna se har'ia evidente en una eventual medici'on.
Teniendo en cuenta esto se decidi'o medir con im'agenes de la siguiente distribuci'on de pixeles:
 \begin{itemize}
 \item{800x600}.
 \item{10x48000}
 \item{11x43636}
\end{itemize}
Para los filtros que trabajan im'agenes blanco y Negro ocurre algo similar a lo anterior solo que cada
pixel esta representado por un byte por lo tanto por cada lectura se procesa de a 16 pixeles. De forma
an'aloga asociamos mejor performance con las im'agenes cuyo anchura en pixeles es un multiplo de 16, y 
una peor performance con las que son coprimas con 16, ya que necesitara una lectura mas por cada columna
de la matriz, mas a'un procesaran algunos datos dos veces. \\
Los \textit{sizes} elegidos para estos caso fueron:\\
\begin{itemize}
 \item{800x600}.
 \item{16x30000}
 \item{17x28235}
\end{itemize}
\subsection{Opciones descartadas}
En un momento se penso ver cual era el rendimiento de los algoritmos con una distribuci'on 
aleatoria de im'agenes, pero decidimos omitirla debido a que era muy costosa y no nos 
daria informaci'on valiosa acerca del comportamiento ya que los \textit{sizes} variables pueden comportarse 
de manera an'aloga por la forma de procesar los datos y no habria una diversidad de comportamiento.\\
Se decidio enfocar los casos de prueba en los \textit{sizes} de im'agen y no tanto en el contenido de 
sus pixeles, vale decir no se buscaron especificamente im'agnes con muchos blancos, grises o negros.   



%\section{Conclusiones}
\label{sec:conclusiones}
Como su nombre lo indica, en esta secci'on hablaremos de las concluciones que pudimos obtener durante el desarrollo del 
trabajo pr'actico,a su vez repasando la hipotesis y conjeturas iniciales y relacion'andola con los resultados de los experimentos.

Como primer concluci'on, en todos los filtros, la implementaci'on en \C \ consume m'as ciclos de reloj que la implementaci'on que usa
tenolog'ia \textit{SSE} a'un en los casos borde. Tal es la diferencia de \textit{performance} que es redituable el costo extra en 
la dificultad de la implementaci'on,\textit{debugging} y adaptaci'on al lenguaje ensamblador.

El alto en p'ixeles de una im'agen es un factor sumamente relevante en el rendimiento de los algoritmos tanto en los casos base,como el de
imagenes altas y anchas, a pesar de que la complejidad teorica es la misma.

La manera de recorrer los ciclos en determinados tipos de filtro pueden reutilizarse como es el caso de las funciones que van de color 
a escala de grises, an'alogamente lo mismo ocurre con las im'agenes en blanco y negro.

Siempre hay un menor porcentaje de penalizaci'on en los filtros que realizan m'as operaciones y accesos a memoria ya que 
no hay tanta diferencia entre reprocesar datos y las operaciones intermedias en cada iteraci'on de un ciclo.

El set de instrucciones fue m'as que suficiente para realizar las tareas pedidas.

Compartivamente, el desarrollo en \ass \ fue la actvidad que m'as tiempo abarc'o, con respecto al desarrollo de este informe y
el desarrollo en \C.

Los casos de prueba seleccionados, reflejaron nuestras conjeturas iniciales y experimentalmente pudimos corroborar 
que la selecci'on fue correcta.

Hubieron cosas que pudimos haber hecho pero por limitaciones de tiempo no tuvimos la oportunidad, entre ellas: alinear la memoria a 16 bytes
y utilizar instrucciones de movimiento alineadas, compilar los filtros en \C \ usando las opciones de optimizaci'on del compilador
gcc e implementar cada filtro en \ass \ sin utilizar \textit{SSE}.  




\end{document}
