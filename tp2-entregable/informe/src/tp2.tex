\documentclass[10pt,a4paper]{article}
\usepackage[latin1]{inputenc}
\usepackage{amsmath}
\usepackage{amsfonts}
\usepackage{amssymb}
\usepackage{array} 
\usepackage{tabularx}
\usepackage[spanish, activeacute]{babel}
\usepackage[small]{caption}
\usepackage[txt]{caratula}
\usepackage{a4wide}
\usepackage{ifthen}
\usepackage[section]{algorithm}
\usepackage{algorithmic}

\makeatletter
\newcommand\textsubscript[1]{\@textsubscript{\selectfont#1}}
\def\@textsubscript#1{{\m@th\ensuremath{_{\mbox{\fontsize\sf@size\z@#1}}}}}
\newcommand\textbothscript[2]{%
  \@textbothscript{\selectfont#1}{\selectfont#2}}
\def\@textbothscript#1#2{%
  {\m@th\ensuremath{%
    ^{\mbox{\fontsize\sf@size\z@#1}}%
    _{\mbox{\fontsize\sf@size\z@#2}}}}}
\def\@super{^}\def\@sub{_}

\catcode`^\active\catcode`_\active
\def\@super@sub#1_#2{\textbothscript{#1}{#2}}
\def\@sub@super#1^#2{\textbothscript{#2}{#1}}
\def\@@super#1{\@ifnextchar_{\@super@sub{#1}}{\textsuperscript{#1}}}
\def\@@sub#1{\@ifnextchar^{\@sub@super{#1}}{\textsubscript{#1}}}
\def^{\let\@next\relax\ifmmode\@super\else\let\@next\@@super\fi\@next}
\def_{\let\@next\relax\ifmmode\@sub\else\let\@next\@@sub\fi\@next}
\makeatother

\newcolumntype{C}{>{\footnotesize\centering\arraybackslash}X}

\newcommand{\xmmW}[2]{
\fontfamily{pcr}\selectfont	
\begin{center}
\begin{tabularx}{400pt}{@{}p{60pt}|C|C|C|C|C|C|C|C|l}
\cline{2-9}
\footnotesize{#1} & #2 &\\
\cline{2-9}
\multicolumn{1}{r@{}}{^{\emph{\tiny 128}}} & \multicolumn{8}{c}{\textit{\scriptsize(word packed)}} & \multicolumn{1}{@{}l}{^{\emph{\tiny 0}}}\\
\end{tabularx}
\end{center}
}

\newcommand{\xmmD}[2]{
\fontfamily{pcr}\selectfont	
\begin{center}
\begin{tabularx}{400pt}{@{}p{60pt}|C|C|C|C|l}
\cline{2-5}
\footnotesize{#1} & #2 &\\
\cline{2-5}
\multicolumn{1}{r@{}}{^{\emph{128}}} & \multicolumn{4}{c}{\textit{\scriptsize(double-word packed)}} & \multicolumn{1}{@{}l}{^{\emph{0}}}\\
\end{tabularx}
\end{center}
}

\newcommand{\ass}{\textit{assembler}}
\newcommand{\C}{\textbf{C}}

% Fancy Header
\usepackage{fancyhdr}
\fancyhf{}
\fancyhead{}
\lhead{Seccion} % texto izquierda de la cabecera
\rhead{\Titulo} % número de página a la derecha
\lfoot{} % texto izquierda del pie
\rfoot{\thepage} 
\pagestyle{fancy}


% Carátula
\materia{Organizaci\'on del Computador II}
\submateria{Segundo Cuatrimestre del 2011}
\titulo{Trabajo Pr\'actico N\'umero 2}
\subtitulo{Filtros de imagen}
\integrante{Celave, Martin}{530/08}{tolacelave@gmail.com}
\integrante{Colombo, Ricardo Gaston}{156/08}{ricardogcolombo@gmail.com}
\integrante{Lores, Fernando}{718/01}{ferlores@gmail.com}
\grupo{Lider espiritual: Angiulino "El chino" Cubrepileta}


% Algoritmos
\floatname{algorithm}{Algoritmo}
\algsetup{indent=2em}

\begin{document}

\maketitle


\section{Introducci'on}

El objetivo de este informe es describir el proceso de como la aplicacion algoritmica de filtros sobre imagenes
puede ser optimizada notablemente(en comparaci'on con ANSI \C) si se aprovecha el modelo de instrucciones SIMD. 

Para ello implementamos siete algoritmos en \C\ y en \ass. Los mismos recorren las im'agenes aplicando diferentes c'alculos. Las implementaciones de alto nivel son bastantes triviales y directas gracias a la posibilidad que nos brinda el lenguaje para acceder a las distintas posiciones de memoria. Sin embargo la gran desventaja que tenemos aqu'i es que accedemos y procesamos los datos de a uno. En cambio, en la implementaci'on de bajo nivel, si bien podemos acceder a una gran cantidad de datos por vez, el acceso a ellos hace que los algortimos a veces se vuelvan ingeniosos. 

En la siguiente secci'on explicaremos como encaramos la implementaci'on de los filtros pedidos. Primero describiremos la estrategia que elegimos para iterar las im'agenes junto con sus variaciones. Luego nos abocaremos a describir la mec'anica que utilizamos en cada algoritmo para procesar los datos. Como dijimos antes, las implementaciones en alto nivel son bastantes directas, por lo que no nos detendremos mucho explic'andolas. El lector puede remitirse driectamente al c'odigo para un mayor detalle de los mismos.

En la secci'on \ref{sec:mediciones} realizaremos mediciones de las distintas implementaciones sobre un conjunto de imagenes dado, comparando y analizando los resultados obtenidos. Es claro inmaginar que deber'ia haber una  gran diferencia en cuanto \textit{performance} y esperamos a lo largo de este informe verificar dicha conjetura. Sin embargo tambien veremos que existen casos menos favorables que otros donde si bien la mejora es grande, comparada con otros casos, no lo es tanto. 

Por 'ultimo en la secci'on \ref{sec:conclusiones} recapitularemos los resultados obtenidos, analizando la relaci'on costo-beneficio de las distintas implementaciones. Ademas comentaremos las principales dificultades encontradas y otras experiencias que obtuvimos durante la realizaci'on de este trabajo pr'actico.
\section{Algoritmos}

\subsection{Iterarando im'agenes}
La primera decisi'on a tomar a la hora de implementar los algoritmos fue como recorreremos las im'agenes. En \C\ optamos por la manera convencional, recorriendo secuencialmente la matriz, leyendo de a p'ixeles y cuando llegamos al final de cada fila salteamos el padding. Es claro que hay maneras mas eficiente pero nos pareci'o apropiado tener esta implementaci'on como base de comparaci'on para todas las mejoras que implementaremos. 

Por otro lado, la implementaci'on en \ass\ fu'e mas ingeniosa ya que al utilizar la tecnolog'ia SSE debiamos, en principio, aprovechar al m'aximo lectura de a 16 bytes. Los problemas se presentan cuando quiero asegurarme de que estoy procesando todos los p'ixeles y que estoy evitando correctamente el \textit{padding}. En este trab'ajo pr'actico procesamos im'agenes a color y en escala de grises. Aunque la estrategia fue la misma, se presentaron sutilies diferencias para estos dos casos. 


\subsubsection{Im'agenes escala de grises}
En este caso cada p'ixel es representado por un byte. La iteraci'on en \C\ es bastante simple (Algoritmo \ref{alg:iteracionC-BN}). En cambio en \ass, cada vez que accedemos a memoria con un registro SSE estamos cargando 16 p'ixeles. Por lo tanto la estrategia ser'a recorrer cada fila realizando sucesivas lecturas hasta que queden menos de 16 p'ixeles para que termine la fila. Para procesar los 'ultimos datos, reposicionamos el 'indice para cargar los 'ultimos pixeles y volvemos a iterar. Es claro que algunos pixeles ser'an recalculados, pero de esta manera podemos reutilizar el c'odigo de las iteraciones normales. Cuando inicio el proceso de la nueva fila, avanzo el 'indice teniendo en cuenta el \textit{padding}. Podemos observar el pseudoc'odigo en el Algoritmo \ref{alg:iteracionASM-BN}. Cabe aclarar que los pseudoc'odigos presentados en el informe son de caracter ilustrativo, con el objetivo de explicar las estrategias utilizadas. Para un pseudoc'odigo mas detallado remitirse a los comentarios del c'odigo \ass.

\begin{algorithm}[h!]
\caption{\sc{iteraci'on en c - escala de grises}}\label{alg:iteracionC-BN}
\begin{algorithmic}[1]
\FOR{$i=0$ to $n$}
	\STATE fila $\leftarrow$ i $*$ row\_size
	\FOR{$pos=0$ to $m$}
		\STATE dst[fila + pos] $\leftarrow$ \textit{procesar}(src[fila + pos])
	\ENDFOR
\ENDFOR
\end{algorithmic}
\end{algorithm}

\begin{algorithm}[h!]
\caption{\sc{iteraci'on en assembler - escala de grises}}\label{alg:iteracionASM-BN}
\begin{algorithmic}[1]
\FOR{$i=0$ to $n$}
	\STATE fila $\leftarrow$ i $*$ row\_size
	\FOR{$pos=0$ to $m$}
		\STATE dst[fila + pos] $\leftarrow$ \textit{procesar_{16}}(src[fila + pos])
		\STATE pos $\leftarrow$ pos + 16		
		\IF{pos $=$ m}
			\STATE \textit{procesar\_siguiente\_fila}
		\ELSIF{pos $>$ m}		 
			\STATE pos $\leftarrow$ w - 16
		\ENDIF
		
	\ENDFOR
\ENDFOR
\end{algorithmic}
\end{algorithm}

\subsubsection{Im'agenes a color}
En este caso cada pixel de las imagenes de entrada esta representado por tres bytes(BGR). La 'unica modificacion que sufre la implementacion en \C\ es que a la hora de acceder a la matriz de entrada multiplicamos el iterador de la columna por tres. Ademas podemos acceder a los distintos colores RGB desplazandonos cero, una o dos posiciones a partir de esa posicion (Algoritmo \ref{alg:iteracionC-RGB}).


\begin{algorithm}[h!]
\caption{\sc{iteraci'on en c - color}}\label{alg:iteracionC-RGB}
\begin{algorithmic}[1]
\FOR{$i=0$ to $n$}
	\STATE fila_s $\leftarrow$ i $*$ src\_row\_size	
	\STATE fila_d $\leftarrow$ i $*$ dst\_row\_size
	\FOR{$pos=0$ to $m$}
		\STATE dst[fila_d + pos] $\leftarrow$ \textit{procesar}(src[fila_s + 3 $*$ pos])
	\ENDFOR
\ENDFOR
\end{algorithmic}
\end{algorithm}

En \ass (Algoritmo \ref{alg:iteracionASM-RGB}), las lecturas de a 16 bytes nos dejan en los registros 5 p'ixeles completos mas el primer byte del siguiente. Aqui decidimos descartar ese 'ultimo byte, procesar los datos y escribir en la imagen de salida solamente los resultados obtenidos. Es decir que mientras la imagen de entrada es recorrida de a 15 bytes(5 p'ixeles), la imagen de salida es recorrida de a 5 bytes. Por este motivo resulta imperioso llevar dos iteradores de posicion.

Tambi'en es necesario ajustar la ultima iteraci'on de cada fila. Cuando la lectura de los 'ultimos 16 bytes, estos no quedan cargados como queremos. Para poder procesar los 'ultimos 5 p'ixeles con las mismas instrucciones que tenemos en el ciclo debemos realizar un desplazamiento a derecha de un byte (Figura \ref{est:ciclo}).
Por otro lado, una vez que realizamos la ultima iteraci'on, debemos ajustar la condicion de corte. Como estamos avanzando de a 15 bytes, vamos a haber finalizado la columna cuando la diferencia entre el iterador y el ancho total en bytes de la imagen de entrada sea igual a uno. Es facil de ver que esto solo sucede cuando se reajusta el 'indice, ya que no existe un caso donde se cumpla esa condici'on sin haber pasado antes por el reajuste porque la cantidad de bytes es siempre m'ultiplo de tres. 

\begin{algorithm}[h!]
\caption{\sc{iteraci'on en assembler - color}}\label{alg:iteracionASM-RGB}
\begin{algorithmic}[1]
\FOR{$i=0$ to $n$}
	\STATE fila_s $\leftarrow$ i $*$ src\_row\_size	
	\STATE fila_d $\leftarrow$ i $*$ dst\_row\_size
	\FOR{$pos=0$ to $m$}
		\STATE dato_{$16$} $\leftarrow$ src[fila + pos]
		\STATE dst[fila + pos] $\leftarrow$ \textit{procesar_{16}}(dato)
		\label{alg:iteracionASM-RGB:entry}
		\STATE pos $\leftarrow$ pos + 15		
		\IF{pos + 1 $=$ m}
			\STATE \textit{procesar\_siguiente\_fila}
		\ELSIF{pos $>$ m}		 
			\STATE pos $\leftarrow$ w - 16
			\STATE dato_{$16$} $\leftarrow$ desplazar_{1b}(src[fila + pos])
			\STATE \textbf{goto} \ref{alg:iteracionASM-RGB:entry}
		\ENDIF		
	\ENDFOR
\ENDFOR
\end{algorithmic}
\end{algorithm}


\begin{figure}[hb]
\xmmW{src[15$*$i]}{XR_4 & G_4B_4 & R_3G_3 & B_3R_2 & G_2B_2 & R_1G_1 & B_1R_0 & G_0B_0}
\xmmW{src[3*w-16]}{R_4G_4 & B_4R_3 & G_3B_3 & R_2G_2 & B_2R_1 & G_1B_1 & R_0G_0 & B_0R_{-1}}
\caption{reajuste de carga en la 'ultima iteraci'on de imagenes a color}
\label{est:ciclo}
\end{figure}


\subsection{Separar canales}
El objetivo de este filtro es generar tres im'agenes distintas en escala de grises. Cada imagen debe contener en cada pixel lo correspondiente al valor de la capa que estamos separando del pixel original. Por lo tanto debemos aislar cada valor RGB de manera que posteriormente podamos manipularlos. En general esto se hace con instrucciones de desempaquetado. Sin embargo en nuestro caso preferimos utilizar otra estrategia. La idea es realizar desplazamientos (empaquetados de words) a izquierda y derecha para desempaquetar todos los valores. En la Figura \ref{est:separar-1} podemos observar como con un desplazamiento a derecha de un byte (linea b) podemos desempaquetar ciertos datos, mientras que con un desplazamiento primero a izquierda y luego a derecha, logramos separar los valores que estaban faltando (linea c).

\begin{figure}[ht]
\xmmW{a) src[15$*$i]}{XR_4 & G_4B_4 & R_3G_3 & B_3R_2 & G_2B_2 & R_1G_1 & B_1R_0 & G_0B_0}
\xmmW{b) psrlw 1}{0X& 0G_4 & 0R_3 & 0B_3 & 0G_2 & 0R_1 & 0B_1 & 0G_0}
\xmmW{c) psllw 1, psrlw 1}{0R_4 & 0B_4 & 0G_3 & 0R_2 & 0B_2 & 0G_1 & 0R_0 & 0B_0}
\caption{Estrategia de desempaquetamiento im'agenes a color}
\label{est:separar-1}
\end{figure}

Una vez desempaquetados estos datos, preparamos m'ascaras para colocar ceros en los words que no necesitamos. Si prestamos atenci'on en como quedan conformados los registros desempaquetados notamos que a veces vamos a necesitar filtrar tres words y a veces dos words. Es por eso que solo dos mascaras (desplazandolas a izquierda y derecha) fueron suficientes para separar todos los B, luego los R y por 'utimo los G. Las m'ascaras estan compuestas por words de \textit{unos} donde estan los datos que queremos y ceros donde estan los valores que no necesitamos en este momento. En la Figura \ref{est:separar-2} podemos observar como quedan los registros despues de utilizar las mascaras para filtrar los valores G (lineas d y e) y como queda combinado en un solo registro todos los valores que necesitamos (linea f).

\begin{figure}[ht]
\xmmW{d) pand b, mask_1}{00 & 00 & 0R_3 & 00 & 00 & 0R_1 & 00 & 00}
\xmmW{e) pand c, mask_2}{0R_4 & 00 & 00 & 0R_2 & 00 & 00 & 0R_0 & 00}
\xmmW{f) por d, e}{0R_4 & 00 & 0R_3 & 0R_2 & 00 & 0R_1 & 0R_0 & 00}
\caption{filtrado y combinaci'on usando m'ascaras \sc{\scriptsize 0000FF0000FF0000} y \sc{\scriptsize FF0000FF0000FF00}}
\label{est:separar-2}
\end{figure}

Ahora resta reacomodar los datos para poder escribirlos en memoria. Para esto usamos las instrucciones de intercambio que el procesador nos ofrece: \textit{pshufd, pshuflw y pshufhw}. Con estas tres instrucciones es posible realizar las tres funciones mas comunes de intercambio a nivel de words: \textit{broadcast}, \textit{swap}, \textit{reverse}\footnote{Intel^{\copyright} 64 and IA-32 Architectures Optimization Reference Manual, secci'on 5.4.11}. La Figura \ref{est:separar-3} muestra como reacomodamos los words (lineas g y h) y empaquetamos para obtener en los bytes menos significativos los cuatro pixeles ya procesados (linea i). En este punto solo nos queda escribir en la imagen destino esos 4 bytes, desplazar el registro SSE de modo que el quinto valor quede en la parte menos significativa, y tambi'en escribirlo. Cabe destacar que para hacer esta 'ultima escritura ultilizamos un registro de prop'osito general, porque no existe una instrucci'on que nos permita copiar un byte determinado del registro SSE a memoria.

\begin{figure}[ht]
\xmmW{g) pshufld f, (3,0,2,1)}{0R_4 & 00 & 0R_3 & 0R_2 & 00 & 00 & 0R_1 & 0R_0}
\xmmD{h) pshufld g, (3,1,2,0)}{0R_4  00 & 00  00 & 0R_3  0R_2 & 0R_1  0R_0}
\xmmW{i) packuswb h }{xx & xx & xx & xx & 0R_4 & 00 & R_3R_2 & R_1R_0}
\caption{\textit{swap} de words y double-words}
\label{est:separar-3}
\end{figure}

El proceso de los otros canales son similares. Siempre utilizamos la misma estrategia: separamos los valores, los enmascaramos, los combinamos, y por ultimo realizamos los intercambios necesarios para poder bajarlos a memoria. Para mayor detalle de como procesamos los canales B y G remitirse a los comentarios del c'odigo adjunto.



\subsection{Monocromatizar}
\subsubsection{Monocromatizar \protect{$\epsilon = 1$} }
El objetivo de la funcion monocromatizar es convertir una imagen a escala de grises, en la imagen fuente
cada pixel se compone de 3 bytes,uno por cada componente de color \textbf{RGB}. En su respectiva implementacion en 
\C \ la imagen se recorre linealmente como una matriz, guardando los valores de cada componente de color
que tienen los pixeles y aplic'andole una serie de operaciones que es la siguiente:
\textbf{(R+2*G+B)/4}, finalmente esto es lo que se almacena en la imagen destino.

La idea en \ass es similar solo que, aprovechando la capacidad de almacenamiento de los registros de 128 bits
y teniendo en cuenta que en cada uno de estos registros entran 16 bytes, podemos procesar de hasta 5 pixeles
por cada lectura de memoria, en lugar de 1. Inicialmente cuando leemos en memoria de la imagen fuente, cargamos 16 bytes en un registro de xmm dejando 
las componentes de color como muestra la figura \ref{est:m-uno}:

\begin{figure}[hb]
\xmmW{texto}{B_5R_4 & G_4B_4 & R_3G_3 & B_3R_2 & G_2B_2 & R_1G_1 & B_1R_0 & G_0B_0}
%\xmmD{texto}{0G_2 0B_2 & R_1G_1 & B_1R_0 & G_0B_0}
\caption{como se levanta de memoria normalizar\_asm}
\label{est:m-uno}
\end{figure}
Como puede observarse se levanta un tercio del sexto pixel, si bien esto no nos preocupa ya que procesaremos
hasta el quinto, debemos reacomodar los datos para poder realizar el copiado de manera c'omoda, pero
antes debemos realizar los c'alculos necesarios. Para reacomodarlos  copiamos en otro registros
lo leido en memoria y los corrimos de manera conveniente como muestra la figura \ref{est:m-dos}: 
\begin{figure}[hb]
\xmmW{carga}{B_5R_4 & G_4B_4 & R_3G_3 & B_3R_2 & G_2B_2 & R_1G_1 & B_1R_0 & G_0B_0}
\xmmW{}{ 00B_5&R_4G_4 & B_4R_3 & G_3B_3 & R_2G_2 & B_2R_1 & G_1B_1 & R_0G_0}
\xmmW{}{ 0000 & B_5R_4 & G_4B_4 & R_3G_3 & B_3R_2 & G_2B_2 & R_1G_1 & B_1R_0}
%\xmmD{texto}{0G_2 0B_2 & R_1G_1 & B_1R_0 & G_0B_0}
\caption{como se corren los datos de manera favorable}
\label{est:m-dos}
\end{figure}

Tras analizar distintas opciones llegamos a la conclusion de
que no es posible  hacer las cuentas sin antes desempaquetar los datos ya que de lo contrario se perderia
presici'on o  podria saturar alguna suma, no obstante dado que la division por 4 es la 'ultima operacion que 
se aplica el resultado final es expresable en un byte. Teniendo en cuenta eso procedimos de la siguiente 
manera: 

Por cada uno de los registros mencionados, desempaquetamos tomando como segundo operando un registro de 128
bits con ceros, y los dividimos en partes bajas y altas, como muestra la figura \ref{est:m-tres} : 

\begin{figure}[hb]
\xmmW{xmm}{ 0_0G_2  & 0_0B_2 & 0_0R_1 & 0_0G_1 & 0_0B_1 & 0_0R_0 & 00G_0 & 00B_0}
\xmmW{xmm}{ 00 B_5  & 00 R_4 & 00 G_4 & 00B_4 & 00 R_3 & 00 G_3 & 00 B_3 & 00 R_2}
\caption{desempaquetado}
\label{est:m-tres}
\end{figure}

Para multiplicar y dividir usamos las instrucciones \textit{psrlw} y \textit{psllw} para realizar un corrimiento 
de bits a derecha e izquierda respectivamente en cada word del registro. \\
Una vez realizados los c'alculos necesarios notemos que tenemos las cuentas necesarias en dos registros
y separadas cada dos bytes con lo cual debemos proceder a unirlos para luego copiarlos. \\
El proceso para unirlos consiste en dos partes, una es aplicar una m'ascara, debido a que no podemos 
fiarnos del contenido de los bytes que saparan cada dato que necesitamos, esa m'ascara tendra unos 
y ceros ya que al aplicarle \textit{pand} me quedan los datos necesarios separados por ceros.\\
El siguiente diagrama(figura\ref{est:m-cuatro} )  muestra la parte antes mencionada(c representa los calculos pedidos):\\
\begin{figure}[hb]
\xmmW{datos1}{ 00 & c_2 & 00 & 00 & c_1 & 00 & 00 & c_0}
\xmmW{mascara1}{ 00 & FF & 00 & 00 & FF & 00 & 00 & FF}
\xmmW{datos2}{ 00 & 00 & 00 & c_4 & 00 & 00 & c_3 & 00}
\xmmW{mascara2}{ FF & 00 & 00 & FF & 00 & 00 & FF & 00}
\caption{datos luego de hacer un and con las mascaras}
\label{est:m-cuatro}
\end{figure}
La otra parte consiste en sumarlos una vez aplicada la m'ascara y utilizar las instruciones que provee
la arquitectura para intercambiar datos dentro de un registro(las mismas que se mencionan en
\textbf{separar canales}) con el fin de colocarlos de manera contigua en dos registros y luego empaquetar. Si bien 
estas instrucciones pueden afectar la performance son realmente  'utiles e
indispensables para obtener la funcionalidad deseada(figura \ref{est:m-cinco}).
\begin{figure}[hb]
\xmmW{datos_sumados}{ 00 & c_2 & 00 & c_4 & c_1 & 00 & c_3 & c_0}
\xmmW{intercambio}{ 00 & 00 & c_4 & c_3 & c_2 & c_1 & c_0}
\caption{datos luego de hacer un and con las mascaras}
\label{est:m-cinco}
\end{figure}

Una vez terminado este proceso, copiamos un \textit{dword} a memoria y con eso cubrimos 4 pixeles,
luego hacemos un corrimiento del registro donde tenemos guardados los datos ya calculados para luego,
copiar el quinto pixel restante, incrementamos los punteros adecuadamente para continuar el 
proceso en las proximas iteraciones.\\


\subsubsection{Monocromatizar \protect{$\epsilon = \infty $} }
El objetivo de esta funcion es similar al anterior, solo que esta vez en lugar de aplicar operaciones
una vez acomodados los datos solo debemos obtener el m'aximo.\\
En un principio el procedimiento es similar al de  $\epsilon= 1$, es decir leemos de memoria 16 bytes
y realizamos copias de esa lectura en otros registros para correrlos adecuadamente(a un registro lo corremos un byte 
 y a el otro dos, para luego operar en paralelo y obtener la componente m'axima). Ahora bien, 
notemos que contamos con una ventaja con respecto al caso anterior y es que no necesitamos desempaquetar
ya que no realizamos operaciones de ning'un tipo que puedan resultar en una saturaci'on o 
resultado erroneo, m'as bien nos piden la m'axima componente de color de cada pixel. Luego de analizar 
varias opciones que recurrian m'ascaras de bits, o cuentas auxliliares optamos por utilizar la instrucci'on
\textit{pmaxub} que calcula en m'aximo byte a byte entre dos registros xmm, esto no solo facilita 
el desarrollo sino que adem'as incrementa la performance, ya que esta provista por la arquitectura y 
esta optimizada. Entonces tendriamos algo asi como en la figura \ref{est:m-seis} (denomino m a los m'aximos):\\
\begin{figure}[hb]
\xmmW{}{xxxx & xxm_4 & xx & m_3xx & xxm_2 & xx & m_1xx & xxm_0}
\caption{datos una vez aplicados los sucesivo pmaxub}
\label{est:m-seis}
\end{figure}
Una vez obtenidos los m'aximos de cada pixel(solo nos interesan 5 de ellos), debemos concentrarnos en
tenerlos en un registro de proposito general de manera contigua. 
Copiamos este registro en otro ya que operaremos con ambos y podemos perder ciertos datos. Realizamos un 
corrimiento de 8 bits a izquierda en cada \textit{word} para obtener ceros a la izquierda de cada una de las partes bajas de
cada \textit{word}. En la otra copia hacemos algo parecido solo que hacemos un corrimiento a derecha y luego a izquierda
dejando los registros de la siguiente manera(figura \ref{est:m-siete}) : \\
\begin{figure}[hb]
\xmmW{corrimiento byte a izq}{xx00& m_400 & xx00 & xx00 & m_200 & xx00 & xx00 & m_000 }
\xmmW{corrimiento byte a der}{m_500 & xx00 & xx00 & m_300 & xx00 & xx00 & m_100 & xx00 }
\caption{corrimientos para generar ceros}
\label{est:m-siete}
\end{figure}
Una vez logrado esto debemos proceder a deshacernos de los bytes que no sabemos que contienen y separan
cada uno de los m'aximos, para ello los transformamos en ceros con una m'ascara y luego hacemos un \textit{por} 
asi obtenemos los m'aximos en un solo registro(aunque esten desordenados) como muestra la figura \ref{est:m-ocho}.\\
\begin{figure}[hb]
\xmmW{datos1}{0000& m_400 & 0000 & 0000 & m_200 & 0000 & 0000 & m_000 }
\xmmW{mascara1}{ 00 & FF & 00 & 00 & FF & 00 & 00 & FF}
\xmmW{datos2}{m_500 & 0000 & 0000 & m_300 & 0000 & 0000 & m_100 & 0000 }
\xmmW{mascara2}{ FF & 00 & 00 & FF & 00 & 00 & FF & 00}
\xmmW{or}{m5_00& m_400 & 0000 & m_300 & m_200 &0000 & m_100 & m_000 }
\caption{datos luego de hacer un and con las mascaras y or}
\label{est:m-ocho}
\end{figure}
Nuevamente ordenamos los contenidos de los registros utilizando las instrucciones de intercambio de bits
dentro de un registro mencionadas previamente, empaquetamos con un registro lleno de ceros
 y de forma an'aloga realizamos el proceso de copiado a la imagen destino. La idea es que luego de
hacer los corrimientos y los empaquetados los datos queden as'i como muestra la figura \ref{est:m-nueve}:\\
\begin{figure}[hb]
\xmmW{datos1}{0000& 0000 & m_500 & m_400 & m_300 & m_200 & m_100 & m_000 }
\xmmW{datos1}{00& 00 & m_5 & m_4 & m_3 & m_2& m_1 & m_0 }
\caption{datos luego de hacer un and con las mascaras y or}
\label{est:m-nueve}
\end{figure}




 
 


\subsection{Umbralizar}
El fin de esta funcion es obtener una imagen en tres colores con lo cual cada pixel se va a procesar observando el valor del mismo y dependiendo de este se devuelve un pixel en la imagen resultante de color negro, gris o blanco, comparando con un umbral maximo y otro minimo.

Esto se define con la siguiente funcion.

I_out(p)=\begin{Bmatrix} 0 & \mbox{ si }& p \leq umbral_min \\128 & \mbox{si}& d \\c & \mbox{si}& p \gt umbral_maximo \end{matrix}


Para realizar este algoritmo primero vimos que era necesario tener los umbrales con lo cual primero definimos el umbral minimo colocando en un registro de proposito general el umbral minimo que recibimos como parametro de entrada, luego shifteamos 8 bits a la izquierda y volvimos a sumar el umbral minimo repitiendo este procedimiento 3 veces mas, con lo cual en el registro nos quedo el umbral minimo repetido 5 veces.
Despues movemos al registro xmm5 el registro de proposito general con el umbral minimo y utilizamos la instruccion para realizar un shufle de a double words y con lo que nos queda repetido el umbral en las posiciones del registro sse. 
Repetimos esta operacion para el umbral maximo colocandolo en el registro xmm4 de la misma manera que lo hicimos con el umbral minimo.

Por otro lado necesitabamos una mascara donde este el numero 128 que representa el gris en la imagen, Para el cual primero colocamos el numero 128 repetido 4 veces en un registro de proposito general,  este es definido en notacion hexadecimal como 80h. Luego colocamos este en el registro xmm6 y con la operacion pshufd lo repetimos este elemento en todas las otras double words contenidas dentro del registro.

\begin{figure}[ht]
\xmmW{paso 5.}{Umin & Umin & Umin & Umin & Umin & Umin & Umin & Umin}
\xmmW{paso 4.}{Umax & Umax & Umax & Umax & Umax & Umax & Umax & Umax}
\xmmW{paso 6.}{128 & 128 & 128 & 128 & 128 & 128 & 128 & 128}
\end{figure}


Una vez ya definidos los umbrales en los registros comenzamos a procesar los datos, primero levantamos la linea en el registro xmm0 , copiamos la mascara del umbral maximo en xmm3 y para saber cuales eran los elementos que estaban sobre el umbral restamos y saturamos a la copia del umbral el valor que tenemos en la linea de los pixeles leidos y con la funcion pcmpeqb con un registro vacio nos queda los numeros hexadecimales F en las posiciones donde es igual a cero , esto quiere decir que el umbral maximo en esas posiciones es superior al valor que se tiene en la imagen, quedandonos F en las pociones que estan sobre el umbral en el registro xmm3.

Siguiente a esto realizamos la diferencia saturada entre los valores en los pixeles de la imagen y un registro que tiene el umbral minimo, con lo cual nos quedan ceros en los bytes que estan debajo del umbral minimo gracias a la saturacion, y con la funcion pcmpeqb nos queda una F en los bytes que estan debajo del umbral en el registro xmm0.

Luego realizamos un OR con los dos registros calculados (xmm0, xmm3) y al resultado le aplicamos la funcion pcmpeqb con un registro vacio para obtener F donde haya ceros, de esta manera esas F representan los pixeles que se encuentran en el medio del umbral.

Luego se le aplica la operacion AND a el resultado con la mascara que se tienen el valor que implicaria los grises, de esta manera nos queda el valor 128 en las pociones intermedias del umbral, y al realizarle la operacion OR con el registro xmm3 que contiene los extremos superiores obtenemos 255 que representa el Blanco en las posiciones que estan sobre el umbral maximo, una vez realizado esto procedemos a guardar el dato.


\subsection{Invertir}
En este filtro, dada una im'agen en escala de grises, debemos calcular la resta entre el valor m'aximo de un p'ixel y el valor de la imagen, es decir, calcular la resta entre 255 y el p'ixel dado.

La implementaci'on en \C\ es mas que trivial, ademas del ciclo ya explicado (con su correspondiente procesamiento lineal de a byte), es solo una resta lo que realiza la funci'on.

Por otra parte en \ass\ lo primero que realizamos es cargar un registro con el valor 255 en cada byte. Para esto usando cualquier registro aplicamos la instrucci'on \textit{pcmpeqb} contra si mismo, dejando todos bits en uno, con lo cual el registro queda completo del valor deseado. Luego realizamos la resta entera saturada \textit{psubb} (aunqe podria ser tambien la no saturada), asi de esta manera obtenemos los valores procesados.

\subsection{Suavizado Gaussiano}
El objetivo de este filtro es obtener una imagen resultante con reducci'on de ruido y difuci'on. 
Para ello utilizaremos el m'etodo de aplicaci'on de m'ascaras, con lo cual el p'ixel resultante se obtiene con la suma
de los valores de los p'ixeles del entorno multiplicados por la m'ascara.

La m'ascara que utilizaremos sera la siguiente(tabla \ref{tab:s-uno} ):

\begin{table}[h!]
\begin{center}
\begin{tabular}{| c | c | c |}
\hline
1/6 & 2/6 & 1/6 \\ \hline
2/6 & 4/6 & 2/6 \\ \hline
1/6 & 2/6 & 1/6 \\ \hline
\end{tabular}
\end{center}
\caption{???????????????????????????????????}
\label{tab:s-uno}
\end{table}

Con lo cual el p'ixel resultante sera el p'ixel que se encuentra en el centro del entorno, y el mismo tendra el siguiente valor:

Al ver que cada p'ixel deb'ia ser la suma de su entorno y que nosotros trabajar'iamos con los registros sse de 128 bits, decidimos 
dividir en 3 partes el algoritmo para tratar cada una de las lineas de la matriz resultante.

Comenzamos el algoritmo levantando de la imagen la primer linea 3 veces pero corriendonos 1 p'ixel en la segunda y 2 p'ixeles en la 
tercer oportunidad, qued'ando de la siguente manera \ref{est:s-dos}.

\begin{figure}[h!]
\xmmW{paso 1.}{a_15a_14 & a_13a_12 & a_11a_10 & a_9a_8 & a_7a_6 & a_5a_4 & a_3a_2 & a_1a_0}
\xmmW{paso 2.}{a_16a_15 & a_14a_13 & a_12a_11 & a_10a_9 & a_8a_7 & a_6a_5 & a_4a_3 & a_2a_1}
\xmmW{paso 3.}{a_17a_16 & a_15a_14 & a_13a_12 & a_11a_10 & a_9a_8 & a_7a_6 & a_5a_4 & a_3a_2}
\xmmW{paso 7}{FF & FF & FF & FF & FF & FF & FF & FF}
\caption{???????????????????????????????????}
\label{est:s-dos}
\end{figure}

Luego desempaquetamos los datos para operar de a words, con lo cual primero copiamos los registros que tenemos con los datos para operar todos los p'ixeles que levantamos, de esta manera nos queda xmm1 \= xmm4 , xmm2 \= xmm5 , xmm3 \= xmm6.
Ya copiados los registros procedimos a desempaquetarlos, previamente habiendo limpiado el registro xmm7 y nos queda \ref{est:s-tres} :

\begin{figure}[h!]
\xmmW{paso 4.}{0 a_15 & 0 a_14 & 0 a_13 & 0 a_12 & a_11 & 0 a_10 & 0 a_9 & 0 a_8}
\xmmW{paso 1.}{0 a_7 & 0 a_6 & 0 a_5 & 0 a_4 & 0 a_3 & 0 a_2 & 0 a_1 & 0 a_0}
\xmmW{paso 5.}{0 a_16 & 0 a_15 & 0 a_14 & 0 a_13 & 0 a_12 & 0 a_11 & 0 a_10 & 0 a_9}
\xmmW{paso 2.}{0 a_8 & 0 a_7 & 0 a_6 & 0 a_5 & 0 a_4 & 0 a_3 & 0 a_2 & 0 a_1}
\xmmW{paso 6.}{0 a_17 & 0 a_16 & 0 a_15 & 0 a_14 & 0 a_13 & 0 a_12 & 0 a_11 & 0 a_10 }
\xmmW{paso 3.}{0 a_9 & 0 a_8 & 0 a_7 & 0 a_6 & 0 a_5 & 0 a_4 & 0 a_3 & 0 a_2}
\xmmW{paso 7}{FF & FF & FF & FF & FF & FF & FF & FF}
\caption{los registros 1,2,3 corresponden a partes bajas y los 4,5,6 a partes altasAsm}
\label{est:s-tres}
\end{figure}
Luego se hace un \textit{shift} para la izquierda en los registros 2 y 5, que ser'ia equivalente a multiplicar por 2, 
y sumamos las partes bajas guard'andolas en el xmm0  y las partes altas en una variable local llamada parte\_alta.
Ya procesada la primer linea de la m'ascara procedemos a la segunda, para la cual sumamos el ancho de la i'magen en 
p'ixeles para pararnos en la linea que corresponden a la segunda fila, luego con el mismo m'etodo copiamos y desempaquetamos, 
usando los mismos registros que en la primer linea, a diferencia que esta vez se multiplican los 
registros xmm1, xmm4 , xmm3 y xmm6 por 2 y los registros xmm2 y xmm6 por 4 , sumamos estos a los acumuladores correspondientes.
Por 'ultimo pasamos a la tercer linea de la misma manera que la anterior, levantamos 3 veces de memoria, moviendonos en 1 
p'ixel con cada linea leida y realizamos el mismo proceso que en la primer linea copiando, desempaquetando y multiplicando por 2 los 
registros xmm2 y xmm5, para luego sumar en los acumuladores correspondientes a la parte alta y parte baja.
Al finalizar de sumar todas las lineas procedemos al empaquetado de la parte alta con la parte baja y guardamos los datos.

\subsection{Normalizar}
El objetivo de este algoritmo es que modificar cada pixel para ampliar el rango de valores de toda la imagen. Para esto utilizamos la f'ormula general (ecuaci'on \ref{eq:normalizar}), estableciendo los valores m'aximos y m'inimos posibles para el caso de las imagenes en escala de grises (255 y 0 respectivamente). Luego realizamos algunas operaciones 'algebraicas a fin de poder establecer una constante de multiplicacion que procesar'a cada pixel.

Entonces el algoritmo queda divido en dos partes: la primera busca el m'aximo y m'inimo de la imagen, mientras que la segunda procesa los valores aplicando la ecuaci'on ya descripta

\begin{equation}
\begin{array}{rl}
I_{out}(i,j) &= (I_{in}(i,j) - max) \times \left( \dfrac{a-b}{max-min} \right) + a  \\
\\
I_{out}(i,j) &= (I_{in}(i,j) - max) \times \left( \dfrac{255-0}{max-min} \right) + 255  \\
\\
I_{out}(i,j) &= \dfrac{255 \times (I_{in}(i,j) - min)}{max-min}  $\hspace{10pt} \textit{\scriptsize factor comun (max - min)}$ \\
\\
$Sea $ K &\leftarrow \dfrac{255}{max-min}  \\
\\
I_{out}(i,j) &= K \times (I_{in}(i,j) - min) \\
\\ \\
$Ecuaci'on \theequation:$&$Desarrollo de la f'ormula de normalizado$
\end{array} 
\label{eq:normalizar}
\end{equation}


\subsubsection*{B'usqueda de m'aximo y m'inimo}

Para encontrar los valores m'aximos y m'inimos nos valemos de las instrucciones \textit{pmaxub} y \textit{pminub}. Las mismas comparan los valores byte a byte quedandose con los correspondientes a cada funci'on. Entonces, despues de recorrer la matriz obtenemos dos registros, uno con los candidatos a m'aximo y otro con los candidatos a m'inimo (paso 0). 

Lo que queda por hacer entonces es comparar los bytes dentro de cada registro para identificar al mayor y menor. Para ello duplicamos el registro y realizamos sucesivos intercambios para comparar todos con todos. Iniciamos invirtiendo la parte alta por la parte baja de registro y volviendo a utilizar las instruciones de m'aximo y mi'nimo para comparar (paso 1). Entonces ahora tenemos ocho candidatos en la parte alta del registro. Repetimos el proceso utilizando intercambiando los dos double-words mas significativos (paso 2). Ahora tenemos cuatro candidatos, una vez mas repetimos la operacion intercambiando ahora los words mas significativos (paso 3).

En este punto tenemos dos candidatos en el word mas significativo del registro. Para compararlos, duplicamos el registro y hacemos un desplazamiento a izquierda de un byte (paso 4). De esta manera, al comparar queda en el byte mas significativo el valor buscado. 

Una vez obtenido estos dos valores, hacemos un desplazamiento a izquierda de tres bytes, y en el double word mas significativo nos queda el valor deseado. Ahora hacemos un \textit{broadcast} a todos los double words y el registro nos queda con los valores deseados empaquetados a double word.

La Figura \ref{est:normalizar-1} muestra como se van reduciendo los candidatos aplicando las operaciones antes descriptas.

\begin{figure}[ht]
\xmmW{paso 0.}{cc & cc & cc & cc & cc & cc & cc & cc}
\xmmW{paso 1.}{cc & cc & cc & cc & xx & xx & xx & xx}
\xmmW{paso 2.}{cc & cc & xx & xx & xx & xx & xx & xx}
\xmmW{paso 3.}{cc & xx & xx & xx & xx & xx & xx & xx}
\xmmW{paso 4.}{mx & xx & xx & xx & xx & xx & xx & xx}
\xmmD{paso 5.}{00 0m & 00 0m & 00 0m & 00 0m}
\caption{\textit{busqueda de m'aximo y m'inimo}}
\label{est:normalizar-1}
\end{figure}


\subsubsection*{C'alculo de valores}
%\section{Mediciones}
\label{sec:mediciones}
Para realizar las mediciones utilizamos las herramientas provistas por la c'atedra para la medici'on de ciclos de \textit{clock}, utilizando una computadora con procesador Intel T2500 2.0GHz Core Duo. El objetivo ser'a medir los distintos algoritmos, comparando las distintas versiones para distintos tipos de imagenes de entrada. Para ello creamos un conjunto de \textit{scripts} de \textit{bash} que realizas las distintas mediciones que necesitamos. Los mismos se encuentran en la carpeta mediciones junto con una explicacion de como utilizarlo.

\subsection{Estableciendo Criterios}

Antes de comenzar tuvimos que tomar ciertas decisiones de c'omo ibamos a realizar dichas mediciones. El primer problema que encontramos al utilizar el reloj del procesador para medir es que nuestro proceso puede llegar a ser interrumpido por otros procesos o por el sistema operativo. Hemos tomado dos medidas al respecto, la primera es ejecutar las mediciones estableciendo m'axima la priodidad del proceso. Para ello Linux provee el comando \texttt{nice}, que ejecutandolo como \textit{root}, permite decirle al sistema operativo que nuestras mediciones tienen prioridad sobre los otros programas en ejecuci'on. 
Mientras que la segunda medida fue realizar mil mediciones por cada prueba, promediando el resultado. El n'umero elegido aparece como una soluci'on de compromiso con el tiempo de ejecuci'on de las muestras.

Otro factor a analizar fue que muestras seleccionamos para medir. Debido a la naturaleza de los algoritmos implementados, podemos ver facilmente que la complejidad esta relacionada directamente con el tama'no de la entrada. Es decir que las dos implementaciones de cada filtro se van a comportar diferente dependiendo del tama'no de la im'agen\footnote{Es importante aclarar que consideramos como tama'no de la imagen, la cantidad total de p'ixeles que esta tiene} y no del tipo de imagen a procesar. Por lo tanto podriamos tomar imagenes de distinto tama'no y comparar como se comportan los distintos filtros y las distintas implementaciones, observando las variaciones de \textit{clocks} con respecto a la cantidad de p'ixeles procesados. Sin embargo elegimos otro camino. Decidimos utilizar un tama'no fijo (dentro de los posible) de p'ixeles y analizar el comportamiento de los algoritmos al variar la relaci'on entre la cantidad de filas y de columnas de la imagen. 

Para las mediciones base utilizamos una imagen de 800x600. Es decir que estamos trabajando con im'agenes de 480.000 p'ixeles (aprox. $0,5Mp$). Luego tomamos distintas ima'genes reacomodando las filas y columnas, seg'un criterios que explicaremos en la siguiente secc'on. Sin embargo a veces no fue posible llegar a la misma cantidad exacta de pixeles. En estos casos elegimos combinaciones donde la diferencia sea m'inima y podemos asegurar que en ningun caso la diferencia de tama'no supera un $0,1\%$ del total.

\subsection{An'alisis de factores}
Antes de decidir las combinaciones de filas y columnas, analizamos con que tipo de im'agenes nuestros algoritmos pod'ian
llegar a comportarse una una manera especial. Como conjetura inicial establecimos que no deberia haber 
fluctuaciones en el las funciones desarrolladas en \C \ ya que el recorrido de la matriz es lineal y
de a un elemento por vez, con lo cual las dimensiones no deber'ian ser un factor relevante. 

En cambio, para el caso de las funciones en \ass, podemos observar que su gran ventaja es que pueden procesar de a muchos datos. Sin embargo su desventaja es que si la imagen no es m'ultiplo de la cantidad de p'ixeles que procesa por iteracion, entonces se debe realizar un reajuste (ver secci'on \label{sec:ciclos}), con el objetivo de procesar los 'ultimos p'ixeles de la fila.

Entonces, para elegir exactamente las combinaciones de filas/columnas tenemos que diferenciar dos tipos de
filtros: Los que toman como par'ametro de entrada im'agenes a color (monocromatizar, separar canales) y los que toman im'agenes en blanco y negro
(umbralizar, invertir, normalizar, suavizar).

Para los que toman imagen blanco y negro, los algoritmos procesan de a 16 pixeles por vez, con lo cual si el ancho es multiplo de 16 el algoritmo realizar'ia una cantidad justa de lecturas en memoria (\textit{caso favorable}). Por el contrario, cuando la anchura no es multiplo de 5 el algoritmo debe realizar una lectura extra al final de cada fila para procesar los bytes restantes(\textit{caso desfavorable}). Es posible que esta lectura extra se haga solo para procesar un pixel. Es decir que estariamos recalculando 15 bytes para procesar solo uno. Esta penalizaci'on se ver'ia acrecentada conforme aumenta la cantidad de filas.

En cambio, los algoritmos que procesan im'agenes a color avanzan de a cinco p'ixeles. Por lo que podemos aplicar el mismo razonamiento que para los algoritmos en blanco y negro modificando el factor de multiplicidad a 5.

\subsection{Eleccion de im'agenes}
Como primera medida, realizamos mediciones con una im'agen de 800x600, para poder establecer un punto de comparaci'on para las otras im'agenes de entrada. Luego, teniendo en cuenta lo descripto en la secci'on anterior, elegimos tama'nos de im'agenes que provoquen casos favorables y casos desfavorables, para poder compararlos. 

Para el segundo grupo de mediciones elegimos para los algoritmos en blanco y negro im'agenes de 16 y 17 p'ixeles de ancho para los casos favorable y desfavorable respectivamente. Estos valores son los anchos m'as peque'nos que producen los casos que buscamos. Por el contrario, para los algoritmos a color elegimos anchos de 10 y 11 p'ixeles. Es de esperar que en este conjunto de mediciones la penalizac'ion del \ass se vea exagerada.

Luego realizamos otro conjunto de mediciones con imagenes mas anchas. Intentando, a priori, diluir las penalizaciones cuando las haya. En este punto encontramos un imprevisto. La librer'ia openCV genera un error al intentar abrir im'agenes mas anchas que 1453 p'ixeles. Es por eso que elegimos un ancho lo mas cercano a ese valor y que fuera m'ultiplo de 16 y de 5, de esta manera nos servir'ia como caso favorable tanto para los algoritmos a color como los de blanco y negro. El ancho elegido fue 1440, ya que cumple las condiciones anteriores. Por otra parte elegimos como ancho para los casos desfavorables el valor 1441.


\subsection{Experimentos}
\subsubsection{Caso base}
En este experimento se miden los siguientes 'indices:
\begin{itemize}
 \item \textbf{ciclos}: ciclos de reloj consumidos por cada implementaci'on.
 \item \textbf{diferencia de ciclos}: modulo de la diferencia entre ciclos de reloj en \C \ y en \ass.
 \item \textbf{porcentaje de mejora}: porcentaje de eficiencia de un implementaci'on con respecto a la otra.
\end{itemize}

Podemos observar en el cuadro \ref{tab:base} que en todos los filtros la implementaci'on en \ass \ consume
menos ciclos de reloj, un resultado a que esta dentro de lo esperado, sin embargo llama la atenci'on
lo significativa que es esta diferencia. Por ejemplo en el filtro invertir, que es el que menos instrucciones
realiza, obtuvimos un porcentaje de mejora superior al 96$\%$. El caso del filtro que menos porcentaje
de mejora tuvo fue monocromatizar\_uno, que no obstante, obtuvo una mejora superior al 50$\%$. Si bien
suponiamos que podia haber una diferencia en la eficiencia no sabiamos que seria tan marcada, lo cual es muy 
positivo, ya que significa que dio r'editos implementar en un lenguaje de m'as bajo nivel y mayor dificultad
 a cambio de un mejor rendimiento,siempre utilizando la tecnologia \textit{SSE}.

\begin{table}[h!]
\begin{tabular}{|c|c|c|c|c|c|}
\hline
\sc{funci'on} & \sc{\# pixels }& \sc{ciclos C }& \sc{ciclos ASM }& \sc{$\delta$ ciclos }& \sc{\% mejora}\\ \hline
invertir & 800x600 & 7.317.170 & 287.367 & 7.029.803 & 96,07\%\\ 
normalizar & 800x600 & 23.331.351 & 1.539.564 & 21.791.786 & 93,40\%\\ 
umbralizar & 800x600 & 13.538.943 & 540.008 & 12.998.935 & 96,01\%\\ 
suavizar & 800x600 & 32.483.342 & 3.968.673 & 28.514.669 & 87,78\%\\ 
monocromatizar\_uno & 800x600 & 17.332.554 & 5.513.335 & 11.819.219 & 68,19\%\\ 
monocromatizar\_inf & 800x600 & 18.555.484 & 4.468.378 & 14.087.106 & 75,92\%\\ 
separar\_canales & 800x600 & 20.421.239 & 6.219.616 & 14.201.623 & 69,54\%\\ 
\hline
\end{tabular}
\caption{Resultados caso base}
\label{tab:base}
\end{table}

\subsubsection{Imagenes altas}
A partir de este experimento introdugimos un nuevo concepto de indice denominado
\textbf{penalizaci'on}. Este 'indice tiene como par'ametros de comparaci'on a los ciclos necesarios por
cada implementaci'on en \ass, mide que porcentaje representa la cantidad de ciclos en un ancho 
favorable con respecto a uno no favorable.

Estudiemos primero el caso de los filtros blanco y negro. Como muestra el cuadro \ref{tab:abyn} a'un existe una 
mejora de \ass  con respecto a \C \  en las dimensiones de 16x30000 y 17x28234. Con esto concluimos que aun
en un caso no favorable es \ass \ sigue siendo m'as eficiente.

Otra observacio'n interesante es el porcentaje de penalizaci'on, notemos que dan 340\%, 165\% y
234\%, para invertir, normalizar y umbralizar respectivamente, estos porcentajes son altos 
y seguramente se debe a lo que dijimos en un principio, que el algoritmo debe reprocesar datos m'as de 
una vez cada vez que itera sobre las 'ultima columnas.

\begin{table}[h!]
\begin{tabular}{|c|c|c|c|c|c|c|}
\hline
\sc{funci'on} & \sc{\# pixels} & \sc{ciclos C }& \sc{ciclos ASM }& \sc{$\Delta$ ciclos }& \sc{\% mejora }& \sc{penalizaci'on}\\ \hline
invertir & 16x30000 & 7.706.474 & 408.430 & 7.298.044 & 94,70\% & \\ 
 & 17x28234 & 7.642.414 & 1.799.789 & 5.842.625 & 76,45\% & 340,66\%\\ \hline
normalizar & 16x30000 & 24.321.128 & 1.821.235 & 22.499.894 & 92,51\% & \\ 
 & 17x28234 & 23.895.976 & 4.829.856 & 19.066.120 & 79,79\% & 165,20\%\\ \hline
umbralizar & 16x30000 & 14.287.793 & 680.832 & 13.606.961 & 95,23\% & \\ 
 & 17x28234 & 14.253.437 & 2.275.706 & 11.977.731 & 84,03\% & 234,25\%\\ \hline
suavizar & 17x28234 & 29.283.006 & 3.773.630 & 25.509.376 & 87,11\% & \\ 
 & 18x26666 & 29.320.184 & 7.404.941 & 21.915.243 & 74,74\% & 96,23\%\\ 
\hline
\end{tabular}
\caption{Resultados Im'agenes altas blanco y negro}
\label{tab:abyn}
\end{table}


Haciendo un an'alisis similar para el caso color, 
podemos apreciar en el cuadro \ref{tab:acolor} la diferencia de ciclos en \C \ y \ass \ y su porcentaje de mejora es en todos los casos
alto( mayor a 50\%) y el diferencial es similar en todos los filtros es decir,
la diferencia m'axima entre el porcentaje de caso favorable y caso no favorable es de n 12\% para la funci'on monocromatizar\_uno,
el resto se mantiene abajo del 10\%. Un muy buen indicador de rendimiento ya que hay poca diferencia de mejora de
comportamiento entre un caso favorable y otro no favorable.

Para los casos no favorables es de espera que no haya una similitud con respecto al caso base en cuanto a ciclo de 
reloj(implementacion en \ass ), esto se debe al reprocesaiento de datos. Al igual que en el caso de filtro blanco y negro
hay el porcentaje de mejora de ciclos cay'o, consumiendo mas ciclos en los casos no favorables, si bien esto es una 
observacion trivial, es un buen indicador de que los casos de prueba fueron bien seleccionados.

Deteng'amonos un momento en los 'indices de penalizaci'on, podemos apreciar que los porcentajes son mucho 
menores con respecto al caso de blanco y negro,lo cual significa que el costo de reprocesar de datos en
las funciones de color fue mucho menor que en dichas funciones,esto seguamente se debe a que los filtros de de color
trabajan con pixeles de 3 bytes cada uno, por lo tanto reprocesa menos pixeles.

El 'unico filtro que result'o  llamativo el porcentaje de penalizaci'on es el suavizado, ya que dicho porcentaje es bajo lo cual
nos hace inferir que tiene menos diferencia de comportamiento entre casos favorables y casos no favorables, a diferencia
de los otros filtros donde la penalizaci'on es mayor. Si bien un bajo i'ndce de penalizaci'on podria significar algo bueno,
tambien puede ocurrir que se deba a que en un caso favorable el algoritmo se comporta de manera similar a un caso no favorable
y eso es claramente malo. 

Como una observaci'on general de estas mediciones de imagenes altas en contraste con el caso base podemos apreciar la 
similitud en los ciclos de reloj de las implementaciones en \C \ y las diferencias en \ass \ sobre todo en los 
casos no favorables esto se debe a la distribuci'on de los pixeles.

\begin{table}[h!]
\begin{tabular}{|c|c|c|c|c|c|c|}
\hline
\sc{\footnotesize funci'on} & \sc{\footnotesize\# pixels} & \sc{\footnotesize ciclos C }& \sc{\footnotesize ciclos ASM }& \sc{\footnotesize$\Delta$ ciclos }& \sc{\footnotesize\% mejora }& \sc{\footnotesize penalizaci'on}\\ \hline
mono\_uno& 10x48000 & 18.232.058 & 6.168.852 & 12.063.205 & 66,16\% & \\ 
 & 11x43636 & 18.059.469 & 8.144.680 & 9.914.789 & 54,90\% & 32,03\%\\ \hline
mono\_inf & 10x48000 & 18.978.920 & 5.410.115 & 13.568.806 & 71,49\% & \\ 
 & 11x43636 & 18.967.671 & 6.368.449 & 12.599.222 & 66,42\% & 17,71\%\\ \hline
separar\_c & 10x48000 & 20.874.906 & 7.176.690 & 13.698.216 & 65,62\% & \\ 
 & 11x43636 & 20.800.016 & 9.195.779 & 11.604.237 & 55,79\% & 28,13\%\\ \hline
\end{tabular}
\caption{Resultados Im'agenes altas color}
\label{tab:acolor}
\end{table}


\subsubsection{Imagenes anchas}
Analizemos el cuadro \ref{tab:anchas} de im'agenes anchas, notemos en la tabla \textbf{color alto}, que el porcentaje de mejora es
casi tan alto como el de el caso base, en todos los filtros, esto a simple vista no deber'ia
 sorprendernos pero si observamos m'as atentamente el porcentaje de mejora en los casos desfavorables es 
casi tan alto como en el caso favorable,algo que con \textbf{imagenes altas} no ocurria, por el contrario
deca'ia bastante. Esto es positivo en terminos de performance y probablemente se debe a que si bien
debe reprocesar datos, no son influyentes ya que los aproximadamente 480000 est'an distribuidos en
pocas filas.

Observemos ahora los'indices de penalizaci'on podemos notar que si bien existen (vale decir hay un porcentaje
 de penalizaci'on) este es significativamente menor al de \textbf{imagenes altas} para todos los filtros,esto
deber'ia captar nuesta atenci'on ya que por cada columna en los casos no favorables se reprocesan datos,pero
si lo pensamos dos veces tiene mucho sentido, ya que cuando el ancho era variable, las matrices
tenian m'as filas con lo cual asint'oticamente esa penalizaci'on iria incrementando, ahora las matrices
tienen una cantidad de filas constantes (que encima es comparativamente chica con respecto al ancho),
por lo tanto ese reprocesamiento de datos, si bien no es despreciable, es efectivamente mucho menor,
a tal punto de no afectar a la \textit{performance} del filtro.

Como otra observaci'on general notemos que la cantidad de ciclos consumidos por la implementaci'on en \C \ es
similar al de los casos base, por el lado de la implementaci'on en \ass \ en todos los casos salvo normalizar,
los filtros consumen m'as ciclos que en su caso base.

Otro dato que llama la atenci'on es el filtro de suavizar que mantiene una regularidad en los i'ndices de 
\textbf{diferencia de ciclos} y \textbf{ciclosASM} con respecto al caso base, lo que nos permite concluir que es el filtro que
presento la mayor similitud de rendimiento con a pesar de que la cantidad de filas es casi la mitad con respecto al caso
base.



\begin{table}[h!]
\begin{tabular}{|c|c|c|c|c|c|c|}
\hline
\sc{\footnotesize funci'on} & \sc{\footnotesize\# pixels} & \sc{\footnotesize ciclos C }& \sc{\footnotesize ciclos ASM }& \sc{\footnotesize$\Delta$ ciclos }& \sc{\footnotesize\% mejora }& \sc{\footnotesize penalizaci'on}\\ \hline
invertir & 1440x333 & 7.277.945 & 313.567 & 6.964.378 & 95,69\% &  \\ 
  & 1441x333 & 7.274.547 & 459.617 & 6.814.930 & 93,68\% & 46,58\%\\ \hline
normalizar & 1440x333 & 23.352.742 & 1.557.537 & 21.795.205 & 93,33\% &  \\ 
  & 1441x333 & 23.326.450 & 1.793.687 & 21.532.763 & 92,31\% & 15,16\%\\ \hline
umbralizar & 1440x333 & 13.004.767 & 543.850 & 12.460.917 & 95,82\% &  \\ 
  & 1441x333 & 13.013.609 & 715.497 & 12.298.113 & 94,50\% & 31,56\%\\ \hline
suavizar & 1440x333 & 32.468.271 & 3.905.847 & 28.562.424 & 87,97\% &  \\ 
  & 1441x333 & 32.483.078 & 3.945.091 & 28.537.988 & 87,85\% & 1,00\%\\ \hline
mono\_uno & 1440x333 & 17.292.896 & 5.499.913 & 11.792.983 & 68,20\% &  \\ 
  & 1441x333 & 17.337.919 & 5.529.326 & 11.808.594 & 68,11\% & 0,53\%\\ \hline
mono\_inf & 1440x333 & 18.526.267 & 4.450.636 & 14.075.631 & 75,98\% &  \\ 
  & 1441x333 & 18.521.016 & 4.483.062 & 14.037.954 & 75,79\% & 0,73\%\\ \hline
separar\_c& 1440x333 & 20.427.885 & 6.163.018 & 14.264.867 & 69,83\% &  \\ 
  & 1441x333 & 20.457.241 & 6.214.148 & 14.243.092 & 69,62\% & 0,83\%\\ \hline
\end{tabular}
\caption{Resultados Im'agenes anchas}
\label{tab:anchas}
\end{table}



%\section{Conclusiones}
\label{sec:conclusiones}
\begin{itemize}
\item En todos los filtros y las im'agenes que procesamos la tecnolog'ia \textit{SSE} supera ampliamente la implementaci'on en \C \ en t'erminos de \textit{performance}, a'un en los casos borde. Sin embargo tambi'en ese ahorro tiempo de ejecuci'on se ve traducido en un esfuerzo mucho mayor en la etapa de desarrollo de los filtros. Mientras que cada algoritmo se resolv'ia en \C\ en minutos, la implementaci'on en \ass\ llevaba varias horas y tal vez dias. Es cierto que esto se debe que no estabamos acostumbrados a trabajar con dicha tecnolog'ia, sin embargo no solo el desarrollo fue lento sino tambien fue tedioso y a veces encontramos muchas dificultades a la hora de encontrar errores. Inmaginamos que en ciertos casos tal es la diferencia de \textit{performance} es determinante y por ende es redituable el costo extra de la implementaci'on,\textit{debugging} y adaptaci'on al lenguaje ensamblador.

\item Nuestra conjetura inicial era que la implementaci'on en C deberia ser alrededor de 16 veces mas lenta que la de \ass. Nunca inmaginamos que pudiera ser tan grande la diferencia, tal vez sobreestimamos lo que el compilador de C hace a la hora de compilar. Tal vez utilizando opciones de optimizaci'on nos hubiesemos acercado mas a esa realci'on te'orica.

\item La distribuci'on de p'ixeles dentro de una im'agen puede llegar a ser un factor a tener en cuenta para cuando se aplica la versi'on optimizada. Por el contrario, las implentaciones en \C\ se comportaron de manera estable a lo largo de todos los experimentos.

\item La manera de recorrer las im'agenes en determinados tipos de filtro pudieron reutilizarse como es el caso de las funciones que van de color 
a escala de grises, an'alogamente lo mismo ocurre con las im'agenes en blanco y negro.

\item Siempre hay un menor porcentaje de penalizaci'on en los filtros que realizan m'as operaciones y accesos a memoria ya que 
no hay tanta diferencia entre reprocesar datos y las operaciones intermedias en cada iteraci'on de un ciclo.

\item Seguramente no estamos contemplando cierta informacion, pues si bien los resultados en general daban como esperabamos a veces las diferencias son mas grandes o mas peque'nas de lo que esperabamos o podemos justificar con los elementos que contamos. Creemos que un an'alisis mas profundo, en conjunto con otros experimentos que individualicen otros aspectos del procesador podrian echar mas luz sobre los misterios de la optimizaci'on de bajo nivel. 

\item El set de instrucciones de la verion 2 fue m'as que suficiente para realizar las tareas pedidas.

\item Compartivamente, el desarrollo en \ass \ fue la actvidad que m'as tiempo abarc'o, con respecto al desarrollo de este informe, mediciones y
el desarrollo en \C.

\item Los casos de prueba seleccionados ayudaron a probar nuestras conjeturas iniciales y experimentalmente pudimos corroborar 
que la selecci'on fue correcta.

\item Hubieron experimentos que no pudimos realizar por limitaciones de tiempo pero que nos hubiesen gustado ver, entre ellos: alinear la memoria a 16 bytes para utilizar la instruccion de movimiento alineado y ver las diferencias de \textit{performance}, realizar mediciones contra los filtros en \C \ usando las opciones de optimizaci'on del compilador e implementar cada filtro en \ass \ sin utilizar tecnolog'ia \textit{SSE}.  
\end{itemize}



\end{document}
