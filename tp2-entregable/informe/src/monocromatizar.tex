\subsection{Monocromatizar}
\subsubsection{Monocromatizar \protect{$\epsilon = 1$} }
El objetivo de la funcion monocromatizar es convertir una imagen a escala de grises, en la imagen fuente
cada pixel se compone de 3 bytes,uno por cada componente de color \textbf{RGB}. En su respectiva implementacion en 
\C la imagen se recorre linealmente como una matriz, guardando los valores de cada componente de color
que tienen los pixeles y aplic'andole una serie de operaciones que es la siguiente:
\textbf{(R+2*G+B)/4}, finalmente esto es lo que se almacena en la imagen destino.

La idea en \ass es similar solo que, aprovechando la capacidad de almacenamiento de los registros de 128 bits
y teniendo en cuenta que en cada uno de estos registros entran 16 bytes, podemos procesar de hasta 5 pixeles
por cada lectura de memoria, en lugar de 1. Inicialmente cuando leemos en memoria de la imagen fuente, cargamos 16 bytes en un registro de xmm dejando 
las componentes de color de la siguiente manera: 

\begin{figure}[hb]
\xmmW{texto}{B_5R_4 & G_4B_4 & R_3G_3 & B_3R_2 & G_2B_2 & R_1G_1 & B_1R_0 & G_0B_0}
%\xmmD{texto}{0G_2 0B_2 & R_1G_1 & B_1R_0 & G_0B_0}
\caption{como se levanta de memoria normalizar\_asm}
\label{est:m-uno}
\end{figure}
Como puede observarse se levanta un tercio del sexto pixel, si bien esto no nos preocupa ya que procesaremos
hasta el quinto, debemos reacomodar los datos para poder realizar el copiado de manera c'omoda, pero
antes debemos realizar los c'alculos necesarios. Para reacomodarlos  copiamos en otro registros
lo leido en memoria y los corrimos de manera conveniente como muestra la figura: 
\begin{figure}[hb]
\xmmW{carga}{B_5R_4 & G_4B_4 & R_3G_3 & B_3R_2 & G_2B_2 & R_1G_1 & B_1R_0 & G_0B_0}
\xmmW{}{ 00B_5&R_4G_4 & B_4R_3 & G_3B_3 & R_2G_2 & B_2R_1 & G_1B_1 & R_0G_0}
\xmmW{}{ 0000 & B_5R_4 & G_4B_4 & R_3G_3 & B_3R_2 & G_2B_2 & R_1G_1 & B_1R_0}
%\xmmD{texto}{0G_2 0B_2 & R_1G_1 & B_1R_0 & G_0B_0}
\caption{como se corren los datos de manera favorable}
\label{est:m-uno}
\end{figure}

Tras analizar distintas opciones llegamos a la conclusion de
que no es posible  hacer las cuentas sin antes desempaquetar los datos ya que de lo contrario se perderia
presici'on o  podria saturar alguna suma, no obstante dado que la division por 4 es la 'ultima operacion que 
se aplica el resultado final es expresable en un byte. Teniendo en cuenta eso procedimos de la siguiente 
manera: 

Por cada uno de los registro mencionados desempaquetamos tomando como segundo operando un registro de 128
bits con ceros, y los dividimos en partes bajas y altas, como muestra la figura: 

\begin{figure}[hb]
\xmmW{xmm}{ 0_0G_2  & 0_0B_2 & 0_0R_1 & 0_0G_1 & 0_0B_1 & 0_0R_0 & 00G_0 & 00B_0}
\xmmW{xmm}{ 00 B_5  & 00 R_4 & 00 G_4 & 00B_4 & 00 R_3 & 00 G_3 & 00 B_3 & 00 R_2}
\caption{desempaquetado}
\label{est:m-uno}
\end{figure}

Para multiplicar y dividir usamos las instrucciones \textit{psrlw} y \textit{psllw} para realizar un corrimiento 
de bits a derecha e izquierda respectivamente en cada word del registro. \\
Una vez realizados los c'alculos necesarios notemos que tenemos las cuentas necesarias en dos registros
y separadas cada dos bytes con lo cual debemos proceder a unirlos para luego copiarlos. \\
El proceso para unirlos consiste en dos partes, una es aplicar una m'ascara, debido a que no podemos 
fiarnos del contenido de los bytes que saparan cada dato que necesitamos, esa m'ascara tendra unos 
y ceros ya que al aplicarle \textit{pand} me quedan los datos necesarios separados por ceros.\\
El siguiente diagrama muestra la parte antes mencionada(c representa los calculos pedidos):
\begin{figure}[hb]
\xmmW{datos1}{ 00 & c_2 & 00 & 00 & c_1 & 00 & 00 & c_0}
\xmmW{mascara1}{ 00 & FF & 00 & 00 & FF & 00 & 00 & FF}
\xmmW{datos2}{ 00 & 00 & 00 & c_4 & 00 & 00 & c_3 & 00}
\xmmW{mascara2}{ FF & 00 & 00 & FF & 00 & 00 & FF & 00}
\caption{datos luego de hacer un and con las mascaras}
\label{est:m-uno}
\end{figure}
La otra parte consiste en sumarlos una vez aplicada la mascara y utilizar las instruciones que provee
la arquitectura para correr bytes y dwords dentro de un registro( \textit{pshuflw},\textit{pshufhw},
\textit{pshufd}) con el fin de colocarlos de manera contigua en dos registros y luego enpaquetar. Si bien 
estas instrucciones pueden afectar la performance son realmente de 'utiles e
indispensables para obtener la funcionalidad deseada.
\begin{figure}[hb]
\xmmW{datos_sumados}{ 00 & c_2 & 00 & c_4 & c_1 & 00 & c_3 & c_0}
\xmmW{aplicacion de corrimiento datos internos}{ 00 & 00 & c_4 & c_3 & c_2 & c_1 & c_0}
\caption{datos luego de hacer un and con las mascaras}
\label{est:m-uno}
\end{figure}

Una vez terminado este proceso, copiamos un \textit{dword} a memoria y con eso cubrimos 4 pixeles,
luego hacemos un corrimiento del registro donde tenemos guardados los datos ya calculados para luego,
copiar el quinto pixel restante, incrementamos los punteros adecuadamente para continuar el 
proceso en las proximas iteraciones.\\


\subsubsection{Monocromatizar \protect{$\epsilon = \infty $} }
El objetivo de esta funcion es similar al anterior, solo que esta vez en lugar de aplicar operaciones
una vez acomodados los datos solo debemos obtener el maximo.\\
En un principio el procedimiento es similar al de  $\epsilon= 1$, es decir leemos de memoria 16 bytes
y realizamos copias de esa lectura en otros registros para correrlos adecuadamente(un registro corremos un byte 
 y el otro dos, para luego operar en paralelo y obtener la componente m'axima).  Ahora bien, 
notemos que contamos con una ventaja con respecto al caso anterior y es que no necesitamos desenpaquetar
ya que no realizamos operaciones de ningun tipo que puedan resultar en una saturaci'on o 
resultado erroneo,mas bien nos piden la maxima componente de color de cada pixel. Luego de analizar 
varias opciones utilizando m'ascaras de bits, o cuentas auxliliares optamos por utilizar la instrucciones
\textit{pmaxub} que calcula en laximo byte a byte entre dos registros xmm, esto no solo facilita 
el desarrollo sino que adem'as incrementa la performance ya que esta provista por la arquitectura y 
esta optimizada. Entonces tendriamos algo asi(denomino m a los maximos):\\
\begin{figure}[hb]
\xmmW{}{xxxx & xxm_4 & xx & m_3xx & xxm_2 & xx & m_1xx & xxm_0}
\caption{datos una vez aplicados los sucesivo pmaxub}
\label{est:m-uno}
\end{figure}
Una vez obtenidos los maximos de cada pixel(solo nos interesan 5 de ellos), debemos concentrarnos en
tenerlos en un registro de proposito general de manera contigua. 
Copiamos este registro en otro ya que operaremos con ambos y podemos perder ciertos datos. Realizamos un 
corrimiento de 8 bits a izquierda en cada word para obtener ceros a la izquierda de cada una de las partes bajas de
cada word. En la otra copia hacemos algo parecido solo que hacemos un corrimiento a derecha y luego a izquierda
dejando los registros de la siguiente manera: \\
\begin{figure}[hb]
\xmmW{corrimiento byte a izq}{xx00& m_400 & xx00 & xx00 & m_200 & xx00 & xx00 & m_000 }
\xmmW{corrimiento byte a der}{m_500 & xx00 & xx00 & m_300 & xx00 & xx00 & m_100 & xx00 }
\caption{corrimientos para generar ceros}
\label{est:m-uno}
\end{figure}
Una vez logrado esto debemos proceder a deshacernos de los bytes que no sabemos que contienen y separan
cada uno de los maximos, para ello los transformamos en ceros con una m'ascara y luego hacemos un \textit{por} 
asi obtenemos los maximos en un solo registro( aunque esten desordenados).\\
\begin{figure}[hb]
\xmmW{datos1}{0000& m_400 & 0000 & 0000 & m_200 & 0000 & 0000 & m_000 }
\xmmW{mascara1}{ 00 & FF & 00 & 00 & FF & 00 & 00 & FF}
\xmmW{datos2}{m_500 & 0000 & 0000 & m_300 & 0000 & 0000 & m_100 & 0000 }
\xmmW{mascara2}{ FF & 00 & 00 & FF & 00 & 00 & FF & 00}
\xmmW{or}{m5_00& m_400 & 0000 & m_300 & m_200 &0000 & m_100 & m_000 }
\caption{datos luego de hacer un and con las mascaras y or}
\label{est:m-uno}
\end{figure}
Nuevamente ordenamos los contenidos de los registros utilizando las instrucciones de corrimiento de bits
dentro de un registro mencionadas en la funcion anterior, empaquetamos con un registro lleno de ceros
 y de forma an'aloga realizamos el proceso de copiado a la imagen destino. La idea es que luego de
hacer los corrimientos y los empaqutados los datos queden asi:\\
\begin{figure}[hb]
\xmmW{datos1}{0000& 0000 & m_500 & m_400 & m_300 & m_200 & m_100 & m_000 }
\xmmW{datos1}{00& 00 & m_5 & m_4 & m_3 & m_2& m_1 & m_0 }

\caption{datos luego de hacer un and con las mascaras y or}
\label{est:m-uno}
\end{figure}




 
 

