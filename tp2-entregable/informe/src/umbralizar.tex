\subsection{Umbralizar}
Este filtro proproduce im'agenes con solamente tres colores (blanco, gris y negro), dependiendo del valor que tenga cada p'ixel.
El procesamiento de cada p'ixel esta definido por la siguiente funci'on:

$$I_{out}(p) = \left\{\begin{array}{lcc} 0 & \text{si } p \leq umbralMin\\  128 & \text{si } umbralMin < p \leq umbralMax \\ 255 & \text{si } p > umbralMax \end{array} \right. $$

La implementaci'on en \C\ es nuevamente trival. Recorremos, como siempre en estos casos, secuencialmente de a un byte mientras que el procesamiento de cada p'ixel se reduce a dos comparaciones para elegir el valor correcto del resultado.

En cambio, para la implementaci'on en \ass\ procesaremos de a 16 bytes, y no tendremos que desempaquetar los valores

tenemos varios pasos. Primero vimos que para realizar las comparaciones era necesario tener los umbrales empaquetados(de a bytes) en los registros de 128 bits. Para esto cargamos los valores que recibimos como par'ametro de entrada en un registro de proposito general. Luego realizamos desplazamientos de a de 8 bits a la izquierda y sucesivas sumas del valor hasta que conseguimos un doble \textit{word} lleno. Despues movemos a dos registros de 128 bits el registro de prop'osito general con el umbral m'inimo y m'aximo y utilizamos la instruccion de intercambio \textit{shufle} de double words para replicar el valor a lo largo de todo el registro SSE. 

Por otro lado tambi'en necesitabamos un registro que contenga el n'umero 128, que representa el gris en la im'agen.
Para obtenerlo realizamos el mismo procedimiento que hicimos con los valores m'inimos y m'aximos, solo que el valor inicial lo obtenemos a aprtir de la constante 8080h\footnote{el valor 80h corresponde al valor 128 en decimal} .

El resultado de este proceso se ve en la figura \ref{est:u-uno}.

\begin{figure}[h!]
\xmmW{umbral m'inimo}{Umin & Umin & Umin & Umin & Umin & Umin & Umin & Umin}
\xmmW{umbral m'aximo}{Umax & Umax & Umax & Umax & Umax & Umax & Umax & Umax}
\xmmW{gris}{128 & 128 & 128 & 128 & 128 & 128 & 128 & 128}
\caption{constantes necesarias para el proceso del algoritmo}
\label{est:u-uno}
\end{figure}


Una vez ya definidos los umbrales en los registros comenzamos a procesar los datos. Para saber cuales son los elementos que estan por
sobre el umbral hacemos una resta saturada entre el registro que contiene el umbral m'aximo y los valores originales de la im'agen. De esta manera nos quedan ceros en los valores que eran mayores o iguales al m'aximo y valores distintos a cero en las otras posiciones. Entonces utilizamos 
la instrucci'on \textit{pcmpeqb} para armar una m'ascara a partir de la comparaci'on de este resultado contra un registro lleno de ceros, obteniendo por resultado Fh en las posiciones donde el valor original era mayor al umbral y ceros en las otras posiciones.

Ahora realizo un procedimiento similar. Realizamos una resta saturada es entre el valor original y el valor m'inimo, dando por resultado un registro con ceros en los bytes que estan debajo del umbral m'inimo (nuevamente gracias a la saturacion). Luego, utilizando la instrucci'on \textit{pcmpeqb} contra un registro lleno de ceros, armamos otra m'ascara con Fh en los bytes que estan debajo del umbral m'inimo.

Luego combinamos las dos m'ascaras con un \textit{por} y al resultado le aplicamos la instrucci'on \textit{pcmpeqb} con
un registro con ceros. Esta 'ultima operacion es, en este caso, similar a una negacion bit a bit. Por lo que obtenemos Fh donde hab'ia antes ceros, es decir, en las posiciones donde se encuentran los valores medios de la imagen.

Por 'ultimo, utilizamos esta 'ultima m'ascara junto con el registro que contiene la constante 128 para realizar un \textit{pand} y obtener como resultado un registro con los grises en sus posiciones correspondientes. Por 'ultimo solo queda colocar blancos en las posiciones donde la imagen es mayor al umbral. Recordamos que tenemos calculada una mascara con Fh en esas posiciones y ceros en las demas, por lo que un simple \textit{por} combinaria este registro el con registro con grises, obteniendo los bytes necesarios para escribirlos en la nueva imagen.